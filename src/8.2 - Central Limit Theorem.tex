\documentclass{article}
\usepackage{amsmath}
\usepackage{amssymb}
\usepackage{geometry}

\geometry{margin=1in}

\title{Lecture Summary: Central Limit Theorem (CLT)}
\author{}
\date{}

\begin{document}

\maketitle

\section*{Source: Lec 7.6.pdf}

\section*{Key Points}

\begin{itemize}
  \item \textbf{Definition of CLT:}
    \begin{itemize}
      \item The Central Limit Theorem (CLT) states that for iid random variables $X_1, X_2, \dots, X_n$ with mean $\mu$ and variance $\sigma^2$:
        \[
          Y = \frac{\sum_{i=1}^n (X_i - \mu)}{\sqrt{n}}
        \]
        converges in distribution to the standard normal distribution $N(0, 1)$ as $n \to \infty$.
      \item This result is remarkable because it holds regardless of the original distribution of $X_i$, provided $\mathbb{E}[X_i] = \mu$ and $\text{Var}(X_i) = \sigma^2$.
    \end{itemize}

  \item \textbf{Moment Generating Function (MGF):}
    \begin{itemize}
      \item The MGF for a random variable $X$ is defined as:
        \[
          M_X(\lambda) = \mathbb{E}[e^{\lambda X}].
        \]
      \item Properties:
        \begin{itemize}
          \item $M_X(\lambda)$ determines the distribution of $X$.
          \item If $M_X(\lambda)$ exists for $\lambda$ in an interval around $0$, it can be expanded as:
            \[
              M_X(\lambda) = 1 + \lambda \mathbb{E}[X] + \frac{\lambda^2}{2} \mathbb{E}[X^2] + \cdots,
            \]
            where coefficients correspond to moments of $X$.
        \end{itemize}
    \end{itemize}

  \item \textbf{CLT and Scaling:}
    \begin{itemize}
      \item When the sum $S = \sum_{i=1}^n X_i$ is scaled by $\sqrt{n}$:
        \[
          Y = \frac{S - n\mu}{\sqrt{n\sigma^2}},
        \]
        the resulting distribution approximates $N(0, 1)$.
      \item Scaling by $\sqrt{n}$ rather than $n$ is crucial for obtaining a meaningful distribution in the limit.
    \end{itemize}

  \item \textbf{Applications of CLT:}
    \begin{itemize}
      \item \textbf{Probability Approximations:}
        \begin{itemize}
          \item Example: Binomial$(n, p)$
            \[
              P(Y - n\mu > \delta n\mu) \approx 1 - F\left(\frac{\delta n\mu}{\sqrt{n\sigma^2}}\right),
            \]
            where $F(z)$ is the CDF of $N(0, 1)$.
        \end{itemize}
      \item \textbf{Continuous Distributions:}
        \begin{itemize}
          \item Uniform$[-1, 1]$ with $\mu = 0$ and $\sigma^2 = \frac{1}{3}$:
            \[
              P\left(\frac{Y}{\sqrt{\frac{n}{3}}} > 0.1\sqrt{n}\right) \approx 1 - F(0.1\sqrt{3n}).
            \]
        \end{itemize}
    \end{itemize}

  \item \textbf{Comparison with Weak Law of Large Numbers (WLLN):}
    \begin{itemize}
      \item WLLN:
        \[
          \frac{S}{n} \to \mu \quad \text{in probability as } n \to \infty.
        \]
      \item CLT:
        \[
          \frac{S - n\mu}{\sqrt{n\sigma^2}} \to N(0, 1) \quad \text{in distribution as } n \to \infty.
        \]
      \item CLT provides a distributional approximation, while WLLN focuses on convergence to a constant.
    \end{itemize}
\end{itemize}

\section*{Simplified Explanation}

\textbf{Central Limit Theorem:}
- As sample size increases, the sum (appropriately scaled) of iid random variables approaches a normal distribution, regardless of their original distribution.

\textbf{Why It Matters:}
- Simplifies complex probability calculations.
- Explains why normal distributions appear frequently in real-world data.

\textbf{Applications:}
1. Approximating probabilities for binomial or uniform distributions.
2. Estimating probabilities in high-dimensional or aggregated data.

\section*{Conclusion}

In this lecture, we:
\begin{itemize}
  \item Defined the Central Limit Theorem and its assumptions.
  \item Discussed the importance of scaling by $\sqrt{n}$ for distributional convergence.
  \item Applied CLT to practical examples, demonstrating its utility in probability approximations.
\end{itemize}

The CLT is a cornerstone of probability and statistics, explaining the ubiquity of the normal distribution in diverse contexts.

\end{document}

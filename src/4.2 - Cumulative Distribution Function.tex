\documentclass{article}
\usepackage{amsmath}
\usepackage{amssymb}
\usepackage{geometry}

\geometry{margin=1in}

\title{Lecture Summary: Cumulative Distribution Function (CDF)}
\author{}
\date{}

\begin{document}

\maketitle

\section*{Source: Lecture 4.2.docx}

\section*{Key Points}

\begin{itemize}
  \item \textbf{Definition of Cumulative Distribution Function (CDF):}
    \begin{itemize}
      \item A CDF, denoted as $F_X(x)$, maps any real number $x$ to the interval $[0, 1]$.
      \item Formula:
        \[
          F_X(x) = P(X \leq x).
        \]
      \item The CDF captures the probability that the random variable $X$ takes a value less than or equal to $x$.
    \end{itemize}

  \item \textbf{Properties of CDF:}
    \begin{itemize}
      \item $F_X(x)$ is non-decreasing:
        \[
          \text{If } x_1 \leq x_2, \quad F_X(x_1) \leq F_X(x_2).
        \]
      \item $F_X(x)$ starts at 0 as $x \to -\infty$:
        \[
          \lim_{x \to -\infty} F_X(x) = 0.
        \]
      \item $F_X(x)$ approaches 1 as $x \to \infty$:
        \[
          \lim_{x \to \infty} F_X(x) = 1.
        \]
      \item For $a < b$:
        \[
          P(a < X \leq b) = F_X(b) - F_X(a).
        \]
      \item CDFs for discrete random variables have step-like behavior, while continuous random variables have smooth, continuous CDFs.
    \end{itemize}

  \item \textbf{Example: Bernoulli Random Variable ($X \sim \text{Bernoulli}(p)$):}
    \begin{itemize}
      \item $X$ takes values 0 with probability $1-p$ and 1 with probability $p$.
      \item CDF:
        \[
          F_X(x) =
          \begin{cases}
            0, & x < 0, \\
            1 - p, & 0 \leq x < 1, \\
            1, & x \geq 1.
          \end{cases}
        \]
      \item Graph: Stepwise increase at $x = 0$ and $x = 1$, reflecting the probabilities.
    \end{itemize}

  \item \textbf{Example: Uniform Discrete Random Variable ($X \sim \text{Uniform}\{1, 2, \dots, 6\}$):}
    \begin{itemize}
      \item PMF: $P(X = k) = \frac{1}{6}$ for $k = 1, 2, \dots, 6$.
      \item CDF:
        \[
          F_X(x) =
          \begin{cases}
            0, & x < 1, \\
            \frac{k}{6}, & k \leq x < k+1 \; (k = 1, \dots, 5), \\
            1, & x \geq 6.
          \end{cases}
        \]
      \item Graph: Stepwise increases with each step size corresponding to $\frac{1}{6}$.
    \end{itemize}

  \item \textbf{Applications of CDF:}
    \begin{itemize}
      \item Calculating probabilities for intervals:
        \[
          P(a < X \leq b) = F_X(b) - F_X(a).
        \]
      \item Example:
        \begin{itemize}
          \item $X \sim \text{Uniform}\{1, \dots, 100\}$:
            \[
              P(3 \leq X \leq 10) = F_X(10) - F_X(3) = \frac{10}{100} - \frac{3}{100} = \frac{7}{100}.
            \]
          \item For non-integer values like $3.2$ or $10.6$, the CDF output corresponds to the nearest integer boundary.
        \end{itemize}
      \item Probability for tail events:
        \[
          P(X > c) = 1 - F_X(c).
        \]
    \end{itemize}

  \item \textbf{Connecting CDF to PMF:}
    \begin{itemize}
      \item For discrete random variables:
        \[
          P(X = x) = F_X(x) - F_X(x^-),
        \]
        where $F_X(x^-)$ is the left-hand limit of $F_X$ at $x$.
      \item For continuous random variables, the derivative of $F_X(x)$ yields the probability density function (PDF):
        \[
          f_X(x) = \frac{d}{dx}F_X(x).
        \]
    \end{itemize}
\end{itemize}

\section*{Simplified Explanation}

\textbf{What is the CDF?}
The cumulative distribution function describes the probability that a random variable is less than or equal to a given value:
\[
  F_X(x) = P(X \leq x).
\]

\textbf{Key Features:}
- Non-decreasing.
- Ranges from 0 to 1.
- Stepwise for discrete variables, smooth for continuous ones.

\textbf{Example:}
For $X \sim \text{Bernoulli}(p)$:
- $F_X(0) = 1-p$.
- $F_X(1) = 1$.

\section*{Conclusion}

In this lecture, we:
\begin{itemize}
  \item Defined the CDF and its key properties.
  \item Demonstrated examples with discrete random variables.
  \item Highlighted the use of CDFs in probability calculations and their relationship to PMFs and PDFs.
\end{itemize}

CDFs are crucial for understanding and calculating probabilities in both discrete and continuous frameworks.

\end{document}

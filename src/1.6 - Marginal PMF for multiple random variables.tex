\documentclass{article}
\usepackage{amsmath}
\usepackage{amssymb}
\usepackage{geometry}

\geometry{margin=1in}

\title{Lecture Summary: Marginal PMF for Multiple Random Variables}
\author{}
\date{}

\begin{document}

\maketitle

\section*{Source: Lec1.6.pdf}

\section*{Key Points}

\begin{itemize}
  \item \textbf{Definition of Marginal PMF:}
    \begin{itemize}
      \item The marginal PMF of a random variable $X_1$ in a set of $n$ random variables $X_1, X_2, \dots, X_n$ is given by summing the joint PMF over all other variables:
        \[
          f_{X_1}(t) = \sum_{t_2, t_3, \dots, t_n} f_{X_1, X_2, \dots, X_n}(t, t_2, t_3, \dots, t_n).
        \]
      \item Marginalization simplifies complex joint distributions by focusing on individual variables.
    \end{itemize}

  \item \textbf{Key Principle:}
    \begin{itemize}
      \item To marginalize, keep the variable of interest and sum over the ranges of all other variables.
    \end{itemize}

  \item \textbf{Examples:}
    \begin{itemize}
      \item \textbf{Three Coin Tosses:}
        \begin{itemize}
          \item Experiment: Toss a fair coin three times. Define $X_1, X_2, X_3$ as indicators for heads in each toss.
          \item Joint PMF: $f_{X_1, X_2, X_3}(t_1, t_2, t_3) = \frac{1}{8}$ for all $(t_1, t_2, t_3)$.
          \item Marginal PMF of $X_1$:
            \[
              f_{X_1}(0) = \sum_{t_2, t_3} f_{X_1, X_2, X_3}(0, t_2, t_3) = \frac{1}{2}, \quad f_{X_1}(1) = \frac{1}{2}.
            \]
        \end{itemize}
      \item \textbf{Three-Digit Lottery Numbers:}
        \begin{itemize}
          \item Experiment: Generate a three-digit number.
          \item Variables:
            \begin{itemize}
              \item $X$: First digit (hundreds place).
              \item $Y$: Modulo 2 (even/odd).
              \item $Z$: Last digit (units place).
            \end{itemize}
          \item Marginal PMFs:
            \[
              f_X(x) = \frac{1}{10}, \; x \in \{0, 1, \dots, 9\}, \quad f_Y(y) = \frac{1}{2}, \; y \in \{0, 1\}.
            \]
        \end{itemize}
      \item \textbf{IPL Powerplay Overs:}
        \begin{itemize}
          \item Experiment: Runs scored in six deliveries of the first over.
          \item Random variables: $X_1, X_2, \dots, X_6$.
          \item Marginal PMF for $X_1$ (runs on the first ball):
            \[
              f_{X_1}(0) = \frac{957}{1598}, \; f_{X_1}(1) = \frac{429}{1598}, \; f_{X_1}(4) = \frac{138}{1598}, \; \text{and so on.}
            \]
          \item Marginalization provides meaningful insights even when the joint PMF is too complex to compute.
        \end{itemize}
    \end{itemize}

  \item \textbf{Pairwise Marginalization:}
    \begin{itemize}
      \item The joint PMF of two variables, $X_1$ and $X_2$, is computed by summing over all other variables:
        \[
          f_{X_1, X_2}(t_1, t_2) = \sum_{t_3, \dots, t_n} f_{X_1, X_2, \dots, X_n}(t_1, t_2, t_3, \dots, t_n).
        \]
    \end{itemize}

  \item \textbf{General Formula for Marginal PMFs:}
    \begin{itemize}
      \item For a subset of variables $X_{i_1}, X_{i_2}, \dots, X_{i_k}$:
        \[
          f_{X_{i_1}, X_{i_2}, \dots, X_{i_k}}(t_{i_1}, t_{i_2}, \dots, t_{i_k}) = \sum_{\text{all other variables}} f_{X_1, X_2, \dots, X_n}(t_1, t_2, \dots, t_n).
        \]
    \end{itemize}
\end{itemize}

\section*{Simplified Explanation}

\textbf{Marginal PMFs:}
These focus on individual random variables or subsets by summing over the ranges of other variables.

\textbf{Examples:}
- Coin tosses: $f_{X_1}(0) = f_{X_1}(1) = \frac{1}{2}$.
- IPL runs: Marginal PMFs are derived directly from data proportions.

\textbf{Why Use Marginals?}
They simplify analysis by reducing the complexity of joint distributions.

\section*{Conclusion}

In this lecture, we:
\begin{itemize}
  \item Defined marginal PMFs for multiple random variables.
  \item Explored examples from coin tosses, lottery numbers, and IPL cricket.
  \item Discussed practical use cases for marginalization in large datasets.
\end{itemize}

Marginal PMFs are essential tools for probabilistic analysis and simplifying complex distributions.

\end{document}

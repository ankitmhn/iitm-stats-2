\documentclass{article}
\usepackage{amsmath}
\usepackage{amssymb}
\usepackage{geometry}

\geometry{margin=1in}

\title{Lecture Summary: Parameter Estimation - Concepts and Examples}
\author{}
\date{}

\begin{document}

\maketitle

\section*{Lecture: 9.2 - Introduction to Parameter Estimation}
\section*{Source: Lec8.2.pdf}

\section*{Key Points}

\begin{itemize}
  \item \textbf{Introduction to Parameter Estimation:}
    \begin{itemize}
      \item Parameter estimation is a procedure to determine unknown parameters in a statistical model using data samples.
      \item Involves identifying a function (estimator) that derives the parameter from iid samples.
    \end{itemize}

  \item \textbf{Examples of Parameter Estimation:}
    \begin{enumerate}
      \item \textbf{Bernoulli Trials:}
        \begin{itemize}
          \item Samples: $X_1, X_2, \dots, X_n \sim \text{Bernoulli}(p)$.
          \item Goal: Estimate the success probability $p$ from observed data.
          \item Example Data: $1, 0, 0, 1, 0, 1, 1, 1, 0, 0$ (10 samples).
          \item Larger sample sizes (e.g., 500 samples) improve confidence in the estimate.
        \end{itemize}

      \item \textbf{Radioactive Decay (Poisson Distribution):}
        \begin{itemize}
          \item Model: Number of alpha particles emitted in a fixed time follows $\text{Poisson}(\lambda)$.
          \item $\lambda$ depends on the radioactive substance (e.g., uranium, plutonium).
          \item Data: Observations of particle counts over 2700 intervals.
          \item Goal: Estimate $\lambda$ from the observed counts.
        \end{itemize}

      \item \textbf{Electronic Noise (Gaussian Distribution):}
        \begin{itemize}
          \item Model: Noise in voltage or current follows $N(\mu, \sigma^2)$.
          \item Data: Repeated iid measurements of voltage (e.g., $1.07, 0.91, 0.88$).
          \item Goal: Estimate $\mu$ and $\sigma^2$ from the measurements.
        \end{itemize}
    \end{enumerate}

  \item \textbf{Estimation Process:}
    \begin{itemize}
      \item Estimation involves finding a function (estimator) $\hat{\theta}$ that maps samples to an estimate of the parameter $\theta$.
      \item Example for Bernoulli trials:
        \[
          \hat{p} = \frac{X_1 + X_2 + \cdots + X_n}{n}.
        \]
    \end{itemize}

  \item \textbf{Estimator Properties:}
    \begin{itemize}
      \item The parameter $\theta$ is a constant, but the estimator $\hat{\theta}$ is a random variable dependent on the samples.
      \item Different sample realizations produce different estimates.
      \item Good estimators provide values close to $\theta$ with high probability.
    \end{itemize}

  \item \textbf{Illustrative Example:}
    \begin{itemize}
      \item Estimators for Bernoulli($p$):
        \begin{enumerate}
          \item $\hat{p}_1 = \frac{1}{2}$ (ignores samples).
          \item $\hat{p}_2 = \frac{X_1 + X_2}{2}$ (uses only two samples).
          \item $\hat{p}_3 = \frac{\sum_{i=1}^n X_i}{n}$ (uses all samples).
        \end{enumerate}
      \item $\hat{p}_3$ is more effective as it incorporates all available data.
    \end{itemize}

  \item \textbf{Goals of Parameter Estimation:}
    \begin{itemize}
      \item Design estimators whose distributions are concentrated around the true parameter.
      \item Provide guarantees on the estimator's performance (e.g., bias, variance).
    \end{itemize}
\end{itemize}

\section*{Simplified Explanation}

\textbf{Key Idea:}
Parameter estimation uses sample data to approximate unknown parameters in a statistical model.

\textbf{Examples:}
1. Estimating $p$ in Bernoulli trials.
2. Finding $\lambda$ in radioactive decay models.
3. Determining $\mu$ and $\sigma^2$ for electronic noise.

\textbf{Key Insight:}
Good estimators make full use of sample data to provide reliable approximations of the true parameter.

\section*{Conclusion}

In this lecture, we:
\begin{itemize}
  \item Introduced parameter estimation as a core statistical procedure.
  \item Explored real-world examples involving Bernoulli, Poisson, and Gaussian distributions.
  \item Discussed the design and properties of estimators.
\end{itemize}

Understanding parameter estimation is foundational for statistical modeling and inference.

\end{document}

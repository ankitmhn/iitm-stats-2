\documentclass{article}
\usepackage{amsmath}
\usepackage{amssymb}
\usepackage{geometry}

\geometry{margin=1in}

\title{Lecture Summary: Visualizing Functions of Two Random Variables}
\author{}
\date{}

\begin{document}

\maketitle

\section*{Source: Lecture 2.7.pdf}

\section*{Key Points}

\begin{itemize}
  \item \textbf{Visualization of Functions $g(x, y)$:}
    \begin{itemize}
      \item Functions of two random variables, $g(x, y)$, can be visualized in three dimensions.
      \item A 3D plot places $x$ and $y$ on two axes, and $g(x, y)$ on the third.
      \item While visually appealing, 3D plots are less practical for solving problems.
    \end{itemize}

  \item \textbf{Contour Plots:}
    \begin{itemize}
      \item Contours are 2D visualizations where $g(x, y)$ takes a constant value $c$:
        \[
          g(x, y) = c.
        \]
      \item Example for $g(x, y) = x + y$:
        \begin{itemize}
          \item Contours are straight lines with slope $-1$.
          \item Lines shift based on $c$: $x + y = 0$, $x + y = 2$, $x + y = -2$.
        \end{itemize}
      \item Contour plots are effective for understanding regions and trends in $g(x, y)$.
    \end{itemize}

  \item \textbf{Examples of Contour Functions:}
    \begin{itemize}
      \item \textbf{Sum Function:} $g(x, y) = x + y$
        \begin{itemize}
          \item Contours are straight lines.
          \item Each line represents combinations of $x$ and $y$ summing to $c$.
        \end{itemize}
      \item \textbf{Maximum Function:} $g(x, y) = \max(x, y)$
        \begin{itemize}
          \item Contours are L-shaped lines.
          \item Example: $\max(x, y) = 2$ consists of:
            \begin{itemize}
              \item A vertical line at $x = 2$ for $y < 2$.
              \item A horizontal line at $y = 2$ for $x < 2$.
            \end{itemize}
        \end{itemize}
      \item \textbf{Minimum Function:} $g(x, y) = \min(x, y)$
        \begin{itemize}
          \item Contours are inverse L-shaped.
          \item Example: $\min(x, y) = 2$ includes:
            \begin{itemize}
              \item A vertical line at $x = 2$ for $y > 2$.
              \item A horizontal line at $y = 2$ for $x > 2$.
            \end{itemize}
        \end{itemize}
      \item \textbf{Circle Function:} $g(x, y) = x^2 + y^2$
        \begin{itemize}
          \item Contours are circles with radius $\sqrt{c}$.
          \item Inside the circle, $x^2 + y^2 \leq c$.
        \end{itemize}
    \end{itemize}

  \item \textbf{Regions Below or Above a Value:}
    \begin{itemize}
      \item Visualizing regions where $g(x, y) \leq c$ or $g(x, y) \geq c$ enhances understanding.
      \item Example for $g(x, y) = x + y \leq c$:
        \begin{itemize}
          \item The region below the line $x + y = c$ satisfies the inequality.
        \end{itemize}
    \end{itemize}

  \item \textbf{Applications:}
    \begin{itemize}
      \item Graphing calculators (e.g., Desmos) are useful for plotting contours and regions.
      \item These visualizations aid in understanding and analyzing distributions of two random variables.
    \end{itemize}
\end{itemize}

\section*{Simplified Explanation}

\textbf{What Are Contours?}
Contours are lines or curves where a function $g(x, y)$ takes a constant value. For example:
- $g(x, y) = x + y$: Contours are straight lines.
- $g(x, y) = x^2 + y^2$: Contours are circles.

\textbf{Why Use Contours?}
They help identify patterns and relationships between $x$ and $y$, making problem-solving easier.

\textbf{Example:}
For $g(x, y) = \max(x, y)$ and $g(x, y) = 2$, the contour consists of:
- A vertical line at $x = 2$ for $y \leq 2$.
- A horizontal line at $y = 2$ for $x \leq 2$.

\section*{Conclusion}

In this lecture, we:
\begin{itemize}
  \item Introduced contour plots for visualizing functions of two variables.
  \item Explored examples including sum, max, min, and circle functions.
  \item Highlighted the utility of contours and regions for understanding distributions.
\end{itemize}

Contour visualizations are fundamental tools for analyzing and interpreting relationships in multivariable functions.

\end{document}

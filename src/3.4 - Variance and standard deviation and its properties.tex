\documentclass{article}
\usepackage{amsmath}
\usepackage{amssymb}
\usepackage{geometry}

\geometry{margin=1in}

\title{Lecture Summary: Variance and Its Properties}
\author{}
\date{}

\begin{document}

\maketitle

\section*{Source: Lecture 3.4.pdf}

\section*{Key Points}

\begin{itemize}
  \item \textbf{Motivation for Variance:}
    \begin{itemize}
      \item Expected value provides a central measure but doesn't capture the spread of a random variable.
      \item Example:
        \begin{itemize}
          \item $X = 10$ (constant), $Y \in \{9, 11\}$ (equal probabilities), $Z \in \{0, 20\}$ (equal probabilities).
          \item $E[X] = E[Y] = E[Z] = 10$, but their spreads differ.
        \end{itemize}
    \end{itemize}

  \item \textbf{Definition of Variance:}
    \begin{itemize}
      \item Variance measures the spread of a random variable around its mean:
        \[
          \text{Var}(X) = E[(X - E[X])^2].
        \]
      \item Standard deviation:
        \[
          \sigma(X) = \sqrt{\text{Var}(X)}.
        \]
    \end{itemize}

  \item \textbf{Properties of Variance:}
    \begin{enumerate}
      \item \textbf{Non-Negativity:}
        \[
          \text{Var}(X) \geq 0.
        \]
      \item \textbf{Scaling:}
        \[
          \text{Var}(aX) = a^2 \text{Var}(X).
        \]
      \item \textbf{Translation:}
        \[
          \text{Var}(X + c) = \text{Var}(X).
        \]
      \item \textbf{Variance of the Sum (Independence):}
        \[
          \text{Var}(X + Y) = \text{Var}(X) + \text{Var}(Y), \quad \text{if } X \text{ and } Y \text{ are independent.}
        \]
    \end{enumerate}

  \item \textbf{Alternative Formula for Variance:}
    \begin{itemize}
      \item Variance can also be computed as:
        \[
          \text{Var}(X) = E[X^2] - (E[X])^2.
        \]
    \end{itemize}

  \item \textbf{Examples:}
    \begin{itemize}
      \item \textbf{Die Roll:}
        \begin{itemize}
          \item $X \sim \text{Uniform}\{1, 2, 3, 4, 5, 6\}$.
          \item $E[X] = 3.5$, $\text{Var}(X) = \frac{35}{12}$, $\sigma(X) \approx 1.7078$.
        \end{itemize}

      \item \textbf{Bernoulli Random Variable:}
        \begin{itemize}
          \item $X \sim \text{Bernoulli}(p)$.
          \item $E[X] = p$, $\text{Var}(X) = p(1 - p)$.
        \end{itemize}

      \item \textbf{Binomial Random Variable:}
        \begin{itemize}
          \item $X \sim \text{Binomial}(n, p)$.
          \item $E[X] = np$, $\text{Var}(X) = np(1 - p)$.
        \end{itemize}

      \item \textbf{Poisson Random Variable:}
        \begin{itemize}
          \item $X \sim \text{Poisson}(\lambda)$.
          \item $E[X] = \lambda$, $\text{Var}(X) = \lambda$.
        \end{itemize}
    \end{itemize}

  \item \textbf{Standardization of Random Variables:}
    \begin{itemize}
      \item A random variable $X$ can be standardized as:
        \[
          Y = \frac{X - E[X]}{\sigma(X)},
        \]
        making $E[Y] = 0$ and $\text{Var}(Y) = 1$.
    \end{itemize}

  \item \textbf{Existence of Variance:}
    \begin{itemize}
      \item Variance and expected value may not always exist for certain random variables.
      \item Example: $X = 2^n$ with $P(X = 2^n) = 2^{-n-1}$ leads to divergence of $E[X]$.
      \item Practical cases typically involve well-behaved random variables with finite mean and variance.
    \end{itemize}
\end{itemize}

\section*{Simplified Explanation}

\textbf{Variance:}
Variance measures how spread out a random variable's values are around the mean.
Formula:
\[
  \text{Var}(X) = E[(X - E[X])^2].
\]

\textbf{Key Properties:}
- Scaling: $\text{Var}(aX) = a^2 \text{Var}(X)$.
- Translation doesn't change variance: $\text{Var}(X + c) = \text{Var}(X)$.
- Variance of independent sums: $\text{Var}(X + Y) = \text{Var}(X) + \text{Var}(Y)$.

\textbf{Examples:}
- Die roll: $\text{Var}(X) = \frac{35}{12} \approx 2.916$.
- Poisson random variable: $\text{Var}(X) = \lambda$.

\section*{Conclusion}

In this lecture, we:
\begin{itemize}
  \item Defined variance and its significance in measuring spread.
  \item Derived key properties and an alternative formula for variance.
  \item Standardized random variables for easier comparison.
  \item Highlighted rare cases where variance may not exist.
\end{itemize}

Variance complements expected value by providing a deeper understanding of a random variable's behavior.

\end{document}

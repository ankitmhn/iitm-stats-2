\documentclass{article}
\usepackage{amsmath}
\usepackage{amssymb}
\usepackage{geometry}

\geometry{margin=1in}

\title{Lecture Summary: Simulations and Applications of Expected Value}
\author{}
\date{}

\begin{document}

\maketitle

\section*{Source: Lecture 3.3.pdf}

\section*{Key Points}

\begin{itemize}
  \item \textbf{Expected Value in Practice:}
    \begin{itemize}
      \item The expected value represents the average value of a random variable over a large number of trials.
      \item Simulations are a powerful tool to observe how the expected value aligns with average outcomes in experiments.
    \end{itemize}

  \item \textbf{Casino Dice Game Simulation:}
    \begin{itemize}
      \item Game: A player bets on the outcome of the sum of two dice rolls being under 7, over 7, or equal to 7.
      \item Simulated returns match closely with theoretical expected values:
        \[
          \text{Expected Gain} \approx \text{Simulated Average Gain}.
        \]
      \item Example: For specific betting strategies, simulations consistently show a negative expected gain, reflecting the casino's advantage.
    \end{itemize}

  \item \textbf{Simulating Common Distributions:}
    \begin{itemize}
      \item Python's \texttt{scipy.stats} library generates samples for distributions like Binomial, Geometric, and Poisson.
      \item Simulated averages align closely with theoretical expectations:
        \begin{itemize}
          \item Binomial$(20, 0.3)$: $E[X] = 6$.
          \item Geometric$(0.3)$: $E[X] = \frac{1}{0.3} \approx 3.33$.
          \item Poisson$(\lambda)$: $E[X] = \lambda$.
        \end{itemize}
      \item Highlights the utility of simulations in verifying theoretical results.
    \end{itemize}

  \item \textbf{Balls and Bins Problem:}
    \begin{itemize}
      \item Problem: $m$ balls are thrown into $n$ bins uniformly at random. Compute the expected number of empty bins.
      \item Theoretical result:
        \[
          E[\text{Empty Bins}] = n \left(1 - \frac{1}{n}\right)^m \approx n e^{-\frac{m}{n}}.
        \]
      \item Simulation steps:
        \begin{enumerate}
          \item Repeat the experiment 1000 times (Monte Carlo method).
          \item Assign each ball to a random bin and track the count.
          \item Calculate the average number of empty bins.
        \end{enumerate}
      \item Simulated results closely match theoretical values.
    \end{itemize}

  \item \textbf{Importance of Simulations:}
    \begin{itemize}
      \item Provides a practical way to verify theoretical concepts like expected value.
      \item Demonstrates the law of large numbers: Simulated averages converge to expected values as sample size increases.
    \end{itemize}
\end{itemize}

\section*{Simplified Explanation}

\textbf{Key Idea:}
Simulations show that the average outcomes in repeated trials align with the expected value.

\textbf{Examples:}
- Casino game: Expected gains and simulated gains match closely, highlighting the casino's long-term profit.
- Balls and bins: Simulated and theoretical counts of empty bins are nearly identical.

\textbf{Why It Matters:}
Simulations provide a tangible way to observe and validate probabilistic concepts.

\section*{Conclusion}

In this lecture, we:
\begin{itemize}
  \item Used Python simulations to observe the behavior of expected value.
  \item Verified theoretical results for distributions like Binomial, Geometric, and Poisson.
  \item Demonstrated the practical utility of expected value in games and probabilistic problems.
\end{itemize}

Simulations bridge theory and practice, making abstract concepts like expected value accessible and applicable.

\end{document}

\documentclass{article}
\usepackage{amsmath}
\usepackage{amssymb}
\usepackage{geometry}

\geometry{margin=1in}

\title{Lecture Summary: Visualizing Functions of One Random Variable and One-to-One Functions}
\author{}
\date{}

\begin{document}

\maketitle

\section*{Source: Lecture 2.3.pdf}

\section*{Key Points}

\begin{itemize}
  \item \textbf{Introduction to Functions of Random Variables:}
    \begin{itemize}
      \item Functions of random variables frequently appear in statistical modeling.
      \item Understanding how a function impacts a random variable's distribution is crucial for effective modeling.
    \end{itemize}

  \item \textbf{Visualization of PMFs:}
    \begin{itemize}
      \item Stem plots are a common method for visualizing probability mass functions (PMFs).
      \item Example: For a uniform random variable $X$ on $\{0, 1, \dots, 10\}$, the PMF is:
        \[
          f_X(x) = \frac{1}{11}, \quad x \in \{0, 1, \dots, 10\}.
        \]
      \item Stem plots can also represent distributions like Binomial$(10, 0.5)$, highlighting patterns such as peak probabilities near the mean.
    \end{itemize}

  \item \textbf{One-to-One Functions:}
    \begin{itemize}
      \item A one-to-one function maps each input value to a unique output value.
      \item For such functions, probabilities remain unchanged, but the values are relabeled.
      \item Example 1: $Y = X - 5$ for $X \sim \text{Uniform}(\{0, 1, \dots, 10\})$:
        \[
          Y \in \{-5, -4, \dots, 5\}, \quad P(Y = y) = \frac{1}{11}.
        \]
      \item Example 2: $Y = 2X$ for $X \sim \text{Binomial}(10, 0.5)$:
        \[
          f_Y(y) = f_X(x), \quad y = 2x.
        \]
    \end{itemize}

  \item \textbf{Impact of One-to-One Transformations:}
    \begin{itemize}
      \item Linear transformations such as $Y = aX + b$ result in translated or scaled PMFs.
      \item Stem plots illustrate how transformations affect the spread and alignment of distributions.
    \end{itemize}

  \item \textbf{Monotonicity and Identification:}
    \begin{itemize}
      \item Monotonic functions (increasing or decreasing) are always one-to-one.
      \item A quick test for one-to-one behavior: Check that each horizontal line intersects the function plot at most once.
    \end{itemize}

  \item \textbf{Applications of the Table Method:}
    \begin{itemize}
      \item Create a table of values showing $X$, $Y = f(X)$, and their probabilities.
      \item This method simplifies calculations for transformed distributions, especially for small datasets.
    \end{itemize}
\end{itemize}

\section*{Simplified Explanation}

\textbf{What Happens with One-to-One Functions?}
These functions simply relabel the values of a random variable without altering probabilities. For example:
- $X = 0 \to Y = -5$ under $Y = X - 5$.
- $P(X = x) = P(Y = y)$, with $y = f(x)$.

\textbf{Visualizing Transformations:}
Stem plots help visualize how the PMF shifts or stretches under transformations.

\section*{Conclusion}

In this lecture, we:
\begin{itemize}
  \item Explored one-to-one functions and their effects on random variables.
  \item Highlighted the use of stem plots for visualizing PMFs.
  \item Discussed methods for identifying and working with one-to-one transformations.
\end{itemize}

Understanding one-to-one functions is foundational for analyzing and modeling transformed random variables.

\end{document}

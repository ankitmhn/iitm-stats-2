\documentclass{article}
\usepackage{amsmath}
\usepackage{amssymb}
\usepackage{geometry}

\geometry{margin=1in}

\title{Lecture Summary: Properties of Expected Value}
\author{}
\date{}

\begin{document}

\maketitle

\section*{Source: Lecture 3.2.pdf}

\section*{Key Points}

\begin{itemize}
  \item \textbf{Definition Recap:}
    \begin{itemize}
      \item The expected value, $E[X]$, represents the weighted average of a random variable $X$.
      \item Formula:
        \[
          E[X] = \sum_{t \in T_X} t \cdot P(X = t).
        \]
    \end{itemize}

  \item \textbf{Properties of Expected Value:}
    \begin{enumerate}
      \item \textbf{Expected Value of a Constant:}
        \[
          E[c] = c.
        \]
      \item \textbf{Non-Negativity for Non-Negative Random Variables:}
        \[
          E[X] \geq 0 \quad \text{if } X \geq 0.
        \]
      \item \textbf{Linearity of Expectation:}
        \begin{itemize}
          \item For any random variables $X$ and $Y$, and constants $a, b$:
            \[
              E[aX + bY] = aE[X] + bE[Y].
            \]
          \item This property holds regardless of whether $X$ and $Y$ are independent.
        \end{itemize}
      \item \textbf{Expectation of a Function:}
        \begin{itemize}
          \item For a function $g(X)$:
            \[
              E[g(X)] = \sum_{t \in T_X} g(t) \cdot P(X = t).
            \]
        \end{itemize}
    \end{enumerate}

  \item \textbf{Applications and Examples:}
    \begin{itemize}
      \item \textbf{Casino Math:}
        \begin{itemize}
          \item A betting strategy on outcomes "under 7," "over 7," and "equal to 7" yields:
            \[
              E[\text{Gain}] = \frac{-2 + 3p_1}{6}.
            \]
          \item Result: The expected gain is negative regardless of $p_1$, illustrating how casinos structure bets to ensure long-term losses for players.
        \end{itemize}

      \item \textbf{Linearity of Expectation Example:}
        \begin{itemize}
          \item Sum of two dice rolls:
            \[
              E[X + Y] = E[X] + E[Y] = 3.5 + 3.5 = 7.
            \]
          \item No need to compute the joint distribution of $X$ and $Y$.
        \end{itemize}

      \item \textbf{Binomial Distribution:}
        \begin{itemize}
          \item For $X \sim \text{Binomial}(n, p)$:
            \[
              E[X] = np,
            \]
            derived easily using linearity by summing $n$ independent Bernoulli trials.
        \end{itemize}

      \item \textbf{Centering a Random Variable:}
        \begin{itemize}
          \item To create a random variable with mean 0:
            \[
              Y = X - E[X].
            \]
          \item Applications include machine learning and data normalization.
        \end{itemize}

      \item \textbf{Balls and Bins Problem:}
        \begin{itemize}
          \item Throw 10 balls into 3 bins. Let $X_i = 1$ if bin $i$ is empty, $0$ otherwise.
          \item Expected number of empty bins:
            \[
              E[X_1 + X_2 + X_3] = E[X_1] + E[X_2] + E[X_3] = 3 \cdot \left(\frac{2}{3}\right)^{10}.
            \]
          \item Highlights the simplicity of expectation calculations using linearity.
        \end{itemize}
    \end{itemize}
\end{itemize}

\section*{Simplified Explanation}

\textbf{Key Properties:}
- $E[c] = c$ for a constant $c$.
- $E[X] \geq 0$ if $X \geq 0$.
- $E[aX + bY] = aE[X] + bE[Y]$ simplifies calculations, even for dependent variables.

\textbf{Applications:}
- Casino games ensure negative expected gains for players.
- Simplified expectation computations for the sum of random variables, like dice rolls or binomial distributions.
- Data normalization by centering a random variable: $Y = X - E[X]$.

\section*{Conclusion}

In this lecture, we:
\begin{itemize}
  \item Examined key properties of expected value, including linearity.
  \item Applied these properties to real-world problems such as games, distributions, and data normalization.
  \item Demonstrated the power of expected value in simplifying calculations and gaining insights into probabilistic phenomena.
\end{itemize}

Expected value is an essential tool in probability, offering a balance between theoretical analysis and practical utility.

\end{document}

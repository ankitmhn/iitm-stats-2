\documentclass{article}
\usepackage{amsmath}
\usepackage{amssymb}
\usepackage{geometry}

\geometry{margin=1in}

\title{Lecture Summary: Experiments, Outcomes, Sample Space, and Probability}
\author{}
\date{}

\begin{document}

\maketitle

\section*{Source: Lecture 02.1.pdf}

\section*{Key Points}

\begin{itemize}
  \item \textbf{Core Definitions in Probability:}
    \begin{itemize}
      \item \textbf{Experiment:} Any process or phenomenon that can be studied statistically.
      \item \textbf{Outcome:} The result of an experiment, analogous to the data collected.
      \item \textbf{Sample Space ($S$):} A set containing all possible outcomes of an experiment.
      \item \textbf{Event:} A subset of the sample space that represents outcomes of interest.
      \item \textbf{Probability:} A numerical measure of the likelihood of events in the sample space.
    \end{itemize}

  \item \textbf{Examples of Experiments and Outcomes:}
    \begin{itemize}
      \item \textbf{Coin Toss:}
        \begin{itemize}
          \item Experiment: Tossing a coin.
          \item Outcomes: $\{ \text{Heads}, \text{Tails} \}$.
        \end{itemize}
      \item \textbf{Die Roll:}
        \begin{itemize}
          \item Experiment: Rolling a six-sided die.
          \item Outcomes: $\{ 1, 2, 3, 4, 5, 6 \}$.
        \end{itemize}
      \item \textbf{Complex Experiment: IPL Match:}
        \begin{itemize}
          \item Experiment: Studying an Indian Premier League match.
          \item Outcomes: Could include match results, scores, wickets, ball-by-ball data, etc.
        \end{itemize}
    \end{itemize}

  \item \textbf{Sample Space ($S$):}
    \begin{itemize}
      \item Definition: A mathematical set containing all possible outcomes of an experiment.
      \item Examples:
        \begin{itemize}
          \item Coin Toss: $S = \{\text{Heads}, \text{Tails}\}$.
          \item Die Roll: $S = \{1, 2, 3, 4, 5, 6\}$.
          \item Drawing a Card: $S = \{\text{Suit} \times \text{Value}\}$ for a standard deck.
        \end{itemize}
      \item \textbf{Complexity:} For intricate experiments like IPL matches, the sample space is large and context-dependent, ranging from simple outcomes (match winner) to detailed ball-by-ball data.
    \end{itemize}

  \item \textbf{Toy Examples and Their Importance:}
    \begin{itemize}
      \item Simplified examples, like coin tosses or die rolls, help build theoretical understanding and test hypotheses.
      \item Insights gained from simple models often extend to complex, real-world scenarios.
    \end{itemize}

  \item \textbf{Illustrative Experiments:}
    \begin{itemize}
      \item \textbf{Urn Model:}
        \begin{itemize}
          \item Experiment: Drawing marbles from an urn.
          \item Sample Space: Colors of the marbles.
          \item Variants: Drawing with or without replacement.
        \end{itemize}
      \item \textbf{Pack of Cards:}
        \begin{itemize}
          \item Sample Space: Cartesian product of suits $\times$ values.
          \item Example: Four suits (spades, hearts, diamonds, clubs) and 13 values form 52 cards.
        \end{itemize}
    \end{itemize}

  \item \textbf{Practical Use of Sample Spaces:}
    \begin{itemize}
      \item Writing down the sample space is crucial for clarity when definitions or scenarios are confusing.
      \item Often, practitioners work directly with probabilities without explicitly enumerating the sample space.
    \end{itemize}
\end{itemize}

\section*{Simplified Explanation}

\textbf{Core Ideas:}
An experiment is any process with outcomes that can be studied statistically. The sample space contains all possible outcomes.

\textbf{Examples:}
- Coin Toss: Outcomes are heads or tails.
- Die Roll: Outcomes are numbers 1 to 6.
- Card Draw: Outcomes combine suits and values.

\textbf{Why It Matters:}
Understanding experiments and sample spaces builds the foundation for probability and statistical analysis.

\section*{Conclusion}

In this lecture, we:
\begin{itemize}
  \item Defined experiments, outcomes, and sample spaces.
  \item Explored examples ranging from simple to complex.
  \item Highlighted the practical and theoretical significance of these concepts in probability.
\end{itemize}

These concepts form the basis for more advanced topics in probability and statistics.

\end{document}

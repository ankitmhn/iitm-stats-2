\documentclass{article}
\usepackage{amsmath}
\usepackage{amssymb}
\usepackage{geometry}
\usepackage{graphicx}

\geometry{margin=1in}

\title{Lecture Summary: Continuous Distributions and Their Properties}
\author{}
\date{}

\begin{document}

\maketitle

\section*{Source: Lec 7.7.pdf}

\section*{Key Points}

\begin{itemize}
  \item \textbf{Purpose:}
    \begin{itemize}
      \item Explore key continuous distributions used in statistics.
      \item Focus on their properties, shapes, and interconnections.
      \item Emphasize intuition and visualization over intricate calculations.
    \end{itemize}

  \item \textbf{Parameters of Distributions:}
    \begin{itemize}
      \item Key parameters are categorized as:
        \begin{itemize}
          \item \textbf{Shape Parameters:} Dictate the overall form (e.g., skewness).
          \item \textbf{Location Parameters:} Determine the central tendency.
          \item \textbf{Scale Parameters:} Control the spread or dispersion.
        \end{itemize}
    \end{itemize}

  \item \textbf{Highlighted Distributions:}
    \begin{enumerate}
      \item \textbf{Normal Distribution:}
        \begin{itemize}
          \item PDF:
            \[
              f_X(x) = \frac{1}{\sqrt{2\pi\sigma^2}} e^{-\frac{(x-\mu)^2}{2\sigma^2}}.
            \]
          \item Properties:
            \begin{itemize}
              \item Linear combinations of independent normal variables remain normal.
              \item Applications: Central Limit Theorem, error analysis.
            \end{itemize}
        \end{itemize}

      \item \textbf{Gamma Distribution:}
        \begin{itemize}
          \item PDF (proportional form):
            \[
              f_X(x) \propto x^{\alpha - 1} e^{-\beta x}, \quad x > 0.
            \]
          \item Special case: $\alpha = 1$ gives the exponential distribution.
          \item Applications: Modeling waiting times, rainfall distributions.
        \end{itemize}

      \item \textbf{Beta Distribution:}
        \begin{itemize}
          \item PDF (proportional form):
            \[
              f_X(x) \propto x^{\alpha - 1}(1-x)^{\beta - 1}, \quad 0 < x < 1.
            \]
          \item Properties:
            \begin{itemize}
              \item Mean:
                \[
                  \mathbb{E}[X] = \frac{\alpha}{\alpha + \beta}.
                \]
              \item Variance:
                \[
                  \text{Var}(X) = \frac{\alpha \beta}{(\alpha + \beta)^2 (\alpha + \beta + 1)}.
                \]
            \end{itemize}
        \end{itemize}

      \item \textbf{Cauchy Distribution:}
        \begin{itemize}
          \item PDF:
            \[
              f_X(x) = \frac{1}{\pi \alpha} \frac{\alpha^2}{\alpha^2 + (x-\theta)^2}.
            \]
          \item Properties:
            \begin{itemize}
              \item Undefined mean and variance.
              \item Ratio of two independent normal variables is Cauchy.
            \end{itemize}
        \end{itemize}
    \end{enumerate}

  \item \textbf{Visualization and Applications:}
    \begin{itemize}
      \item Histograms and PDFs help recognize distribution shapes.
      \item Identifying distribution types simplifies modeling tasks in real-world problems.
      \item Examples:
        \begin{itemize}
          \item Normal: Common in measurement errors.
          \item Gamma: Aggregate waiting times.
          \item Beta: Probabilities in a finite range.
          \item Cauchy: Heavy-tailed phenomena.
        \end{itemize}
    \end{itemize}
\end{itemize}

\section*{Simplified Explanation}

\textbf{Key Idea:}
Distributions like Normal, Gamma, Beta, and Cauchy describe various real-world processes. Their parameters dictate shape, location, and spread.

\textbf{Applications:}
- Normal for averages and measurement errors.
- Gamma for waiting times.
- Beta for probabilities in [0, 1].
- Cauchy for heavy tails.

\textbf{Why It Matters:}
Understanding these distributions helps recognize patterns in data and choose appropriate models.

\section*{Conclusion}

In this lecture, we:
\begin{itemize}
  \item Explored common continuous distributions and their properties.
  \item Discussed connections, like how ratios or sums of variables link distributions.
  \item Emphasized visualization and intuition to understand shapes and behaviors.
\end{itemize}

These distributions and their properties form a foundation for advanced statistical modeling and data analysis.

\end{document}

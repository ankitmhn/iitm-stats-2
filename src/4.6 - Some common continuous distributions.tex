\documentclass{article}
\usepackage{amsmath}
\usepackage{amssymb}
\usepackage{geometry}

\geometry{margin=1in}

\title{Lecture Summary: Common Continuous Distributions}
\author{}
\date{}

\begin{document}

\maketitle

\section*{Source: Lecture 4.6.docx}

\section*{Key Points}

\begin{itemize}
  \item \textbf{Introduction:}
    \begin{itemize}
      \item This lecture introduces three common continuous distributions: Uniform, Exponential, and Normal.
      \item These distributions are widely used in modeling practical scenarios and are essential to understand in probability theory.
    \end{itemize}

  \item \textbf{Uniform Distribution:}
    \begin{itemize}
      \item A random variable $X$ is uniformly distributed on $[a, b]$ if its PDF is:
        \[
          f_X(x) =
          \begin{cases}
            \frac{1}{b-a}, & a \leq x \leq b, \\
            0, & \text{otherwise}.
          \end{cases}
        \]
      \item Properties:
        \begin{itemize}
          \item The PDF is flat, indicating equal probability density over $[a, b]$.
          \item CDF:
            \[
              F_X(x) =
              \begin{cases}
                0, & x < a, \\
                \frac{x-a}{b-a}, & a \leq x \leq b, \\
                1, & x > b.
              \end{cases}
            \]
          \item Probabilities are computed as:
            \[
              P(a_1 \leq X \leq b_1) = \frac{b_1 - a_1}{b-a}.
            \]
        \end{itemize}
      \item Example:
        \begin{itemize}
          \item $X \sim \text{Uniform}[-10, 10]$, PDF is $\frac{1}{20}$ for $-10 \leq X \leq 10$.
          \item Probability of $-3 \leq X \leq 2$:
            \[
              P(-3 \leq X \leq 2) = \frac{5}{20} = \frac{1}{4}.
            \]
        \end{itemize}
    \end{itemize}

  \item \textbf{Exponential Distribution:}
    \begin{itemize}
      \item A random variable $X$ follows an exponential distribution with parameter $\lambda > 0$ if its PDF is:
        \[
          f_X(x) =
          \begin{cases}
            \lambda e^{-\lambda x}, & x \geq 0, \\
            0, & x < 0.
          \end{cases}
        \]
      \item Properties:
        \begin{itemize}
          \item The PDF decays exponentially, with most probability density near $x = 0$.
          \item CDF:
            \[
              F_X(x) =
              \begin{cases}
                0, & x < 0, \\
                1 - e^{-\lambda x}, & x \geq 0.
              \end{cases}
            \]
        \end{itemize}
      \item Example:
        \begin{itemize}
          \item $X \sim \text{Exponential}(\lambda=2)$.
          \item Probability of $5 \leq X \leq 7$:
            \[
              P(5 \leq X \leq 7) = \int_5^7 2e^{-2x} \, dx = e^{-10} - e^{-14}.
            \]
        \end{itemize}
      \item Memoryless Property:
        \[
          P(X > s + t \mid X > t) = P(X > s).
        \]
    \end{itemize}

  \item \textbf{Normal (Gaussian) Distribution:}
    \begin{itemize}
      \item A random variable $X$ follows a normal distribution with mean $\mu$ and variance $\sigma^2$ if its PDF is:
        \[
          f_X(x) = \frac{1}{\sqrt{2\pi\sigma^2}} e^{-\frac{(x-\mu)^2}{2\sigma^2}}.
        \]
      \item Properties:
        \begin{itemize}
          \item Symmetric bell-shaped curve centered at $\mu$.
          \item Support: $(-\infty, \infty)$.
          \item CDF does not have a closed form, requiring numerical methods or tables.
        \end{itemize}
      \item Standard Normal Distribution:
        \[
          Z = \frac{X - \mu}{\sigma} \sim \text{Normal}(0, 1).
        \]
      \item Example:
        \begin{itemize}
          \item $X \sim \text{Normal}(\mu=2, \sigma^2=5)$.
          \item Probability of $X < 10$:
            \[
              Z = \frac{10 - 2}{\sqrt{5}} \implies P(X < 10) = \Phi(3.58),
            \]
            where $\Phi$ is the standard normal CDF.
        \end{itemize}
    \end{itemize}
\end{itemize}

\section*{Simplified Explanation}

\textbf{Three Key Distributions:}
1. Uniform: Equal density over an interval.
2. Exponential: Models waiting times; decays exponentially.
3. Normal: Bell-shaped curve; central to many natural phenomena.

\textbf{Applications:}
- Uniform: Modeling equally likely outcomes (e.g., random sampling).
- Exponential: Time until an event (e.g., bus arrival).
- Normal: Modeling data with natural variability (e.g., heights).

\section*{Conclusion}

In this lecture, we:
\begin{itemize}
  \item Introduced the Uniform, Exponential, and Normal distributions.
  \item Defined their PDFs, CDFs, and key properties.
  \item Highlighted practical examples and computation techniques.
\end{itemize}

These distributions are fundamental in probability, appearing frequently in both theory and applications.

\end{document}

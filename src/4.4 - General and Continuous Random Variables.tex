\documentclass{article}
\usepackage{amsmath}
\usepackage{amssymb}
\usepackage{geometry}

\geometry{margin=1in}

\title{Lecture Summary: General and Continuous Random Variables}
\author{}
\date{}

\begin{document}

\maketitle

\section*{Source: Lecture 4.4.docx}

\section*{Key Points}

\begin{itemize}
  \item \textbf{Discrete vs Continuous Random Variables:}
    \begin{itemize}
      \item Discrete random variables are associated with a finite or countably infinite sample space.
      \item Continuous random variables are defined over an interval of real numbers, simplifying modeling for large datasets.
    \end{itemize}

  \item \textbf{Continuous Random Variables:}
    \begin{itemize}
      \item Continuous random variables are modeled using a continuous cumulative distribution function (CDF).
      \item A valid CDF satisfies:
        \begin{enumerate}
          \item Non-decreasing behavior.
          \item Starts at 0 and ends at 1.
          \item Is right-continuous.
        \end{enumerate}
    \end{itemize}

  \item \textbf{Probabilities Using Continuous CDFs:}
    \begin{itemize}
      \item For any interval $a < X \leq b$:
        \[
          P(a < X \leq b) = F_X(b) - F_X(a).
        \]
      \item For a single value $X = c$, the probability:
        \[
          P(X = c) =
          \begin{cases}
            F_X(c) - F_X(c^-), & \text{if there is a jump at } c, \\
            0, & \text{if } F_X(c) \text{ is continuous at } c.
          \end{cases}
        \]
    \end{itemize}

  \item \textbf{Defining Continuous Random Variables:}
    \begin{itemize}
      \item A random variable is continuous if its CDF is continuous at every point.
      \item For continuous random variables:
        \[
          P(X = c) = 0 \quad \forall c.
        \]
    \end{itemize}

  \item \textbf{Examples of CDFs:}
    \begin{itemize}
      \item Discrete random variable: CDF increases in jumps corresponding to probabilities of specific values.
      \item Continuous random variable: CDF rises smoothly with no jumps.
      \item Mixture: Combines discrete jumps and smooth increments.
    \end{itemize}

  \item \textbf{Applications:}
    \begin{itemize}
      \item Scenarios requiring continuous random variables:
        \begin{itemize}
          \item Dartboard distance from the center.
          \item Meteorite weights.
          \item Heights or weights of humans.
          \item Cricket bowling speeds.
          \item Stock prices.
        \end{itemize}
      \item Continuous models are preferred when datasets are large and tracking every discrete value is impractical.
    \end{itemize}
\end{itemize}

\section*{Simplified Explanation}

\textbf{Key Concept: Continuous Random Variables}
Continuous random variables are modeled using a smooth CDF, which simplifies probability calculations for large datasets or ranges.

\textbf{Probabilities with CDFs:}
- Interval: $P(a < X \leq b) = F_X(b) - F_X(a)$.
- Single value: $P(X = c) = 0$ if the CDF is continuous at $c$.

\textbf{Examples:}
- Continuous: Smooth curves (e.g., bowling speeds).
- Discrete: Stepwise jumps (e.g., die rolls).

\textbf{Why Use Continuous Models?}
- Simplifies calculations for datasets with many values.
- Models real-world phenomena efficiently.

\section*{Conclusion}

In this lecture, we:
\begin{itemize}
  \item Distinguished between discrete, continuous, and mixed random variables.
  \item Defined continuous random variables through their CDFs.
  \item Applied CDFs to compute probabilities for different events.
  \item Highlighted the importance of continuous models in practical scenarios.
\end{itemize}

Continuous random variables form a crucial foundation for advanced probability modeling, simplifying calculations and enhancing applicability.

\end{document}

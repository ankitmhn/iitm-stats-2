\documentclass{article}
\usepackage{amsmath}
\usepackage{amssymb}
\usepackage{geometry}

\geometry{margin=1in}

\title{Lecture Summary: Probability Density Function (PDF)}
\author{}
\date{}

\begin{document}

\maketitle

\section*{Source: Lecture 4.5.docx}

\section*{Key Points}

\begin{itemize}
  \item \textbf{Motivation for the PDF:}
    \begin{itemize}
      \item In continuous random variables, the probability mass function (PMF) is replaced by the probability density function (PDF).
      \item A PDF describes the "density" of a random variable over an interval rather than specific probabilities at individual points.
    \end{itemize}

  \item \textbf{Definition of PDF:}
    \begin{itemize}
      \item If $X$ is a continuous random variable with CDF $F_X(x)$, then its PDF $f_X(x)$ is defined as:
        \[
          f_X(x) = \frac{d}{dx}F_X(x).
        \]
      \item The CDF is obtained from the PDF by integration:
        \[
          F_X(x) = \int_{-\infty}^x f_X(t) \, dt.
        \]
    \end{itemize}

  \item \textbf{Key Properties of PDF:}
    \begin{itemize}
      \item $f_X(x) \geq 0$ for all $x$.
      \item The total integral of $f_X(x)$ over its support equals 1:
        \[
          \int_{-\infty}^\infty f_X(x) \, dx = 1.
        \]
      \item The PDF may exceed 1, but it represents probability density per unit length, not the probability itself.
    \end{itemize}

  \item \textbf{Using the PDF for Probabilities:}
    \begin{itemize}
      \item The probability of $X$ falling within an interval $[a, b]$ is given by:
        \[
          P(a \leq X \leq b) = \int_a^b f_X(x) \, dx.
        \]
      \item For any specific value $c$, $P(X = c) = 0$ for continuous random variables.
    \end{itemize}

  \item \textbf{Examples:}
    \begin{itemize}
      \item \textbf{Uniform Distribution:}
        \begin{itemize}
          \item $f_X(x) = \frac{1}{5}$ for $x \in [0, 5]$, $0$ otherwise.
          \item Probability $P(1 \leq X \leq 3)$:
            \[
              P(1 \leq X \leq 3) = \int_1^3 \frac{1}{5} \, dx = \frac{2}{5}.
            \]
        \end{itemize}

      \item \textbf{Triangular Distribution:}
        \begin{itemize}
          \item $f_X(x) = 2x$ for $x \in [0, 1]$, $0$ otherwise.
          \item Verify PDF: Check that $\int_0^1 2x \, dx = 1$.
          \item Probability $P(0.1 \leq X \leq 0.3)$:
            \[
              P(0.1 \leq X \leq 0.3) = \int_{0.1}^{0.3} 2x \, dx = \left[ x^2 \right]_{0.1}^{0.3} = 0.09 - 0.01 = 0.08.
            \]
        \end{itemize}
    \end{itemize}

  \item \textbf{Importance of PDF Over CDF:}
    \begin{itemize}
      \item While the CDF describes cumulative probabilities, the PDF provides a direct view of the distribution's shape and density.
      \item Example: For a bell curve (normal distribution), the PDF highlights the peak around the mean and symmetry, which is harder to deduce from the CDF.
    \end{itemize}
\end{itemize}

\section*{Simplified Explanation}

\textbf{What is a PDF?}
The PDF describes the "density" of a random variable across its range. It replaces the PMF for continuous variables, focusing on intervals instead of specific points.

\textbf{Key Formulae:}
- Probability over an interval:
\[
  P(a \leq X \leq b) = \int_a^b f_X(x) \, dx.
\]
- CDF to PDF:
\[
  f_X(x) = \frac{d}{dx}F_X(x).
\]
- PDF to CDF:
\[
  F_X(x) = \int_{-\infty}^x f_X(t) \, dt.
\]

\textbf{Examples:}
1. Uniform: $f_X(x) = \frac{1}{5}$ for $x \in [0, 5]$.
2. Triangular: $f_X(x) = 2x$ for $x \in [0, 1]$.

\textbf{Why Use PDFs?}
- PDFs offer a clear picture of distribution and density, enabling straightforward calculation of probabilities over intervals.

\section*{Conclusion}

In this lecture, we:
\begin{itemize}
  \item Defined the PDF and its relation to the CDF.
  \item Explored properties and practical use of PDFs for probability calculations.
  \item Illustrated examples of common distributions.
\end{itemize}

The PDF is an essential tool in continuous probability, offering an intuitive and powerful way to analyze distributions.

\end{document}

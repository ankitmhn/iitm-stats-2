\documentclass{article}
\usepackage{amsmath}
\usepackage{amssymb}
\usepackage{geometry}

\geometry{margin=1in}

\title{Lecture Summary: Many-to-One Functions of a Random Variable}
\author{}
\date{}

\begin{document}

\maketitle

\section*{Source: Lec 2.4.pdf}

\section*{Key Points}

\begin{itemize}
  \item \textbf{Introduction to Many-to-One Functions:}
    \begin{itemize}
      \item A many-to-one function maps multiple values of a random variable to the same output value.
      \item Example: $f(x) = (5 - x)^2$ for $x \in \{0, 1, 2, \dots, 10\}$.
      \item This is unlike one-to-one functions, where each input maps to a unique output.
    \end{itemize}

  \item \textbf{Impact on PMFs:}
    \begin{itemize}
      \item When applying a many-to-one function to a random variable:
        \begin{itemize}
          \item Repeated output values occur for different input values.
          \item Probabilities corresponding to repeated values are summed to form the new PMF.
        \end{itemize}
      \item Example:
        \begin{itemize}
          \item If $P(X = 3) = 1/11$ and $P(X = 7) = 1/11$ both map to $f(X) = 4$, then:
            \[
              P(f(X) = 4) = P(X = 3) + P(X = 7) = \frac{2}{11}.
            \]
        \end{itemize}
    \end{itemize}

  \item \textbf{Procedure for Computing PMFs for Many-to-One Functions:}
    \begin{enumerate}
      \item Compute the original PMF of the random variable.
      \item Identify repeated values in the range of the function.
      \item Add the probabilities of all input values mapping to the same output value.
    \end{enumerate}

  \item \textbf{Example Calculation: Uniform Distribution:}
    \begin{itemize}
      \item Original PMF: $X$ uniformly distributed over $\{0, 1, 2, \dots, 10\}$, with $P(X = x) = 1/11$.
      \item Function: $f(X) = (5 - X)^2$.
      \item Output values: $\{0, 1, 4, 9, 16, 25\}$.
      \item New PMF:
        \begin{align*}
          P(f(X) = 0) &= P(X = 5) = \frac{1}{11}, \\
          P(f(X) = 1) &= P(X = 4) + P(X = 6) = \frac{2}{11}, \\
          P(f(X) = 4) &= P(X = 3) + P(X = 7) = \frac{2}{11}, \\
          P(f(X) = 9) &= P(X = 2) + P(X = 8) = \frac{2}{11}, \\
          P(f(X) = 16) &= P(X = 1) + P(X = 9) = \frac{2}{11}, \\
          P(f(X) = 25) &= P(X = 0) + P(X = 10) = \frac{2}{11}.
        \end{align*}
    \end{itemize}

  \item \textbf{Visualization and Interpretation:}
    \begin{itemize}
      \item Stem plots can visualize how the PMF changes under a many-to-one function.
      \item Many-to-one functions often stretch or compress the range of the random variable and alter the PMF significantly.
    \end{itemize}

  \item \textbf{Practical Applications:}
    \begin{itemize}
      \item Many-to-one transformations frequently arise in practice, such as summarizing data or applying mathematical models.
      \item Understanding how functions transform PMFs is essential for statistical modeling and data analysis.
    \end{itemize}
\end{itemize}

\section*{Simplified Explanation}

\textbf{Many-to-One Functions:}
These functions map multiple input values to the same output, requiring probabilities for repeated outputs to be summed.

\textbf{Example:}
For $X$ uniformly distributed over $\{0, 1, \dots, 10\}$ and $f(X) = (5 - X)^2$:
- $P(f(X) = 4) = P(X = 3) + P(X = 7) = 2/11$.

\textbf{Why It Matters:}
This transformation alters the distribution of the random variable, affecting how it is interpreted and analyzed.

\section*{Conclusion}

In this lecture, we:
\begin{itemize}
  \item Explained many-to-one functions and their impact on PMFs.
  \item Derived new PMFs for transformed random variables.
  \item Highlighted the importance of visualization in understanding distribution changes.
\end{itemize}

These concepts are foundational for statistical modeling and practical data analysis.

\end{document}

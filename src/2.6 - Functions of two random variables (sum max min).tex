\documentclass{article}
\usepackage{amsmath}
\usepackage{amssymb}
\usepackage{geometry}

\geometry{margin=1in}

\title{Lecture Summary: Functions of Two Random Variables (Sum, Max, and Min)}
\author{}
\date{}

\begin{document}

\maketitle

\section*{Source: Lec 2.6.pdf}

\section*{Key Points}

\begin{itemize}
  \item \textbf{Introduction to Functions of Two Random Variables:}
    \begin{itemize}
      \item Expands on functions of one random variable to consider combinations of two variables.
      \item Common functions include the sum, maximum, and minimum, which often arise in data analysis.
    \end{itemize}

  \item \textbf{Table Method for Small-Sized Examples:}
    \begin{itemize}
      \item For discrete random variables with few values, the table method remains effective.
      \item Example: $X, Y \sim \text{iid Uniform}\{0, 1\}$, with $Z = X + Y$:
        \begin{itemize}
          \item Joint PMF table:
            \[
              \begin{array}{|c|c|c|c|}
                \hline
                (X, Y) & Z & P(X, Y) \\ \hline
                (0, 0) & 0 & \frac{1}{4} \\
                (0, 1) & 1 & \frac{1}{4} \\
                (1, 0) & 1 & \frac{1}{4} \\
                (1, 1) & 2 & \frac{1}{4} \\ \hline
              \end{array}
            \]
          \item $P(Z = 0) = \frac{1}{4}, \; P(Z = 1) = \frac{1}{2}, \; P(Z = 2) = \frac{1}{4}$.
        \end{itemize}
      \item The method involves identifying repetitions and summing probabilities for each unique value of $Z$.
    \end{itemize}

  \item \textbf{Example for Maximum Function:}
    \begin{itemize}
      \item Define $Z = \max(X, Y)$ with different distributions for $X$ and $Y$.
      \item Tabular computation includes joint PMFs and evaluations of $\max(X, Y)$.
      \item Example:
        \[
          \begin{array}{|c|c|c|c|}
            \hline
            (X, Y) & Z & P(X, Y) \\ \hline
            (0, 0) & 0 & \frac{1}{2} \\
            (0, 1) & 1 & \frac{1}{4} \\
            (1, 0) & 1 & \frac{1}{8} \\
            (1, 1) & 1 & \frac{1}{16} \\
            (0, 2) & 2 & \frac{1}{16} \\ \hline
          \end{array}
        \]
      \item Sum probabilities for repetitions to find:
        \[
          P(Z = 0) = \frac{1}{2}, \quad P(Z = 1) = \frac{11}{32}, \quad P(Z = 2) = \frac{5}{32}.
        \]
    \end{itemize}

  \item \textbf{Challenges for Larger Examples:}
    \begin{itemize}
      \item When $X$ and $Y$ have many possible values, the table method becomes cumbersome.
      \item Example: Two dice rolls produce 36 combinations, manageable but error-prone.
      \item Larger scales, such as $X, Y \in \{1, 2, \dots, 100\}$, yield $10^4$ combinations, making tabular methods impractical.
    \end{itemize}

  \item \textbf{Next Steps for Large Scenarios:}
    \begin{itemize}
      \item Transition to visualization and analytical techniques.
      \item Understanding how functions map $(X, Y)$ to specific outcomes becomes essential for simplification.
    \end{itemize}
\end{itemize}

\section*{Simplified Explanation}

\textbf{Functions of Two Variables:}
Sum, max, and min are common operations. Small datasets can be managed with tables; larger ones require alternative methods.

\textbf{Example:}
For $X, Y \sim \text{Uniform}\{0, 1\}$ and $Z = X + Y$:
- $P(Z = 0) = \frac{1}{4}, \; P(Z = 1) = \frac{1}{2}, \; P(Z = 2) = \frac{1}{4}$.

\textbf{Challenges:}
Tables become impractical for large datasets, necessitating smarter techniques.

\section*{Conclusion}

In this lecture, we:
\begin{itemize}
  \item Applied the table method to functions of two random variables.
  \item Explored examples for sum and max functions.
  \item Discussed limitations of the table method for larger datasets.
\end{itemize}

Future discussions will focus on analytical and visualization approaches for handling large-scale random variable functions.

\end{document}

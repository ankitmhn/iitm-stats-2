\documentclass{article}
\usepackage{amsmath}
\usepackage{amssymb}
\usepackage{geometry}

\geometry{margin=1in}

\title{Lecture Summary: Examples on Functions of a Single Random Variable}
\author{}
\date{}

\begin{document}

\maketitle

\section*{Source: Lec 2.5.pdf}

\section*{Key Points}

\begin{itemize}
  \item \textbf{Applying Functions to Discrete Random Variables:}
    \begin{itemize}
      \item The function maps values of a random variable $X$ to new values, creating a transformed variable $Y$.
      \item A table-based method simplifies this transformation, especially for discrete variables.
    \end{itemize}

  \item \textbf{Example 1: Uniform Random Variable on $\{-5, -4, \dots, 5\}$:}
    \begin{itemize}
      \item $X$ is uniformly distributed over $\{-5, -4, \dots, 5\}$ with $P(X = x) = \frac{1}{11}$.
      \item Function: $Y = f(X)$, where:
        \[
          f(X) =
          \begin{cases}
            0, & X \leq 0, \\
            X, & X > 0.
          \end{cases}
        \]
      \item Table for $Y$:
        \[
          \begin{array}{|c|c|}
            \hline
            Y & P(Y = y) \\ \hline
            0 & \frac{6}{11} \\
            1 & \frac{1}{11} \\
            2 & \frac{1}{11} \\
            3 & \frac{1}{11} \\
            4 & \frac{1}{11} \\
            5 & \frac{1}{11} \\ \hline
          \end{array}
        \]
      \item $Y$ consolidates values of $X \leq 0$ into a single outcome ($Y = 0$).
    \end{itemize}

  \item \textbf{Example 2: Large Uniform Range on $\{-500, -499, \dots, 500\}$:}
    \begin{itemize}
      \item $X$ is uniformly distributed with $P(X = x) = \frac{1}{1001}$.
      \item Function: $Y = \max(X, 5)$:
        \[
          f(X) =
          \begin{cases}
            5, & X \leq 5, \\
            X, & X > 5.
          \end{cases}
        \]
      \item Pattern in $Y$:
        \begin{itemize}
          \item $Y = 5$ occurs for $506$ values of $X$.
          \item $Y = x$ for $x \in \{6, 7, \dots, 500\}$.
        \end{itemize}
      \item Table for $Y$:
        \[
          \begin{array}{|c|c|}
            \hline
            Y & P(Y = y) \\ \hline
            5 & \frac{506}{1001} \\
            6 & \frac{1}{1001} \\
            7 & \frac{1}{1001} \\
            \vdots & \vdots \\
            500 & \frac{1}{1001} \\ \hline
          \end{array}
        \]
    \end{itemize}

  \item \textbf{Key Observations:}
    \begin{itemize}
      \item Using a table to represent transformations helps manage even large datasets.
      \item Patterns in $Y$ can significantly reduce the effort needed to compute PMFs.
    \end{itemize}
\end{itemize}

\section*{Simplified Explanation}

\textbf{Transformations of Random Variables:}
Functions like $f(X)$ consolidate or shift probabilities. Tables simplify deriving new PMFs, especially when patterns exist.

\textbf{Examples:}
- $X \sim \text{Uniform}(-5, 5)$: $f(X)$ maps $X \leq 0$ to $Y = 0$.
- $X \sim \text{Uniform}(-500, 500)$: $f(X) = \max(X, 5)$ creates a concentrated probability at $Y = 5$.

\section*{Conclusion}

In this lecture, we:
\begin{itemize}
  \item Explored how functions transform random variables.
  \item Highlighted the utility of tables for summarizing transformations.
  \item Showed how patterns simplify PMF computations for large datasets.
\end{itemize}

Understanding transformations and patterns enables efficient computation and visualization of transformed distributions.

\end{document}

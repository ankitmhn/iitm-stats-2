\documentclass{article}
\usepackage{amsmath}
\usepackage{amssymb}
\usepackage{geometry}
\usepackage{graphicx}

\geometry{margin=1in}

\title{Lecture Summary: Jointly Continuous Random Variables}
\author{}
\date{}

\begin{document}

\maketitle

\section*{Source: Lecture 5.3.docx}

\section*{Key Points}

\begin{itemize}
  \item \textbf{Introduction:}
    \begin{itemize}
      \item Jointly continuous random variables describe the relationships between two or more continuous variables.
      \item These models are crucial for analyzing phenomena involving simultaneous variation of multiple variables, such as sepal and petal dimensions in the Iris dataset.
    \end{itemize}

  \item \textbf{Joint Density Function:}
    \begin{itemize}
      \item A joint density function $f_{X,Y}(x,y)$ satisfies:
        \begin{enumerate}
          \item $f_{X,Y}(x,y) \geq 0$ for all $(x, y)$.
          \item $\int_{-\infty}^\infty \int_{-\infty}^\infty f_{X,Y}(x,y) \, dx \, dy = 1$.
        \end{enumerate}
      \item The probability of $(X, Y)$ falling within a region $A$ is:
        \[
          P((X, Y) \in A) = \int_A f_{X,Y}(x,y) \, dx \, dy.
        \]
      \item Support is the region where $f_{X,Y}(x,y) > 0$.
    \end{itemize}

  \item \textbf{Uniform Joint Distribution Example:}
    \begin{itemize}
      \item Uniform density in a unit square: $f_{X,Y}(x,y) = 1$ for $0 \leq x, y \leq 1$.
      \item Probability of a subregion is proportional to its area.
      \item Example:
        \begin{itemize}
          \item For region $A$ with $x, y \in [0.1, 0.5]$:
            \[
              P((X, Y) \in A) = \int_{0.1}^{0.5} \int_{0.1}^{0.5} 1 \, dx \, dy = 0.16.
            \]
        \end{itemize}
    \end{itemize}

  \item \textbf{Non-Uniform Joint Density Example:}
    \begin{itemize}
      \item Example density: $f_{X,Y}(x,y) = x + y$ for $0 \leq x, y \leq 1$.
      \item Validation:
        \[
          \int_0^1 \int_0^1 (x + y) \, dx \, dy = 1.
        \]
      \item Probability of region $A$: $x, y \in [0, 0.5]$:
        \[
          P((X, Y) \in A) = \int_0^{0.5} \int_0^{0.5} (x + y) \, dx \, dy = 0.125.
        \]
    \end{itemize}

  \item \textbf{Visualization:}
    \begin{itemize}
      \item 2D and 3D plots help visualize joint densities, showing support and density variations.
      \item Example: Uniform density forms a flat "plate" over the unit square; non-uniform density creates slopes.
    \end{itemize}

  \item \textbf{Applications:}
    \begin{itemize}
      \item Useful for modeling correlated continuous phenomena, such as:
        \begin{itemize}
          \item Sepal length and width in flowers.
          \item Temperature and humidity.
        \end{itemize}
      \item Simplifies probability calculations for regions of interest.
    \end{itemize}
\end{itemize}

\section*{Simplified Explanation}

\textbf{Key Idea:}
Jointly continuous random variables describe two or more variables with a shared density.

\textbf{Uniform Example:}
For $f_{X,Y}(x,y) = 1$ over a unit square, probability is the area of the region.

\textbf{Non-Uniform Example:}
For $f_{X,Y}(x,y) = x + y$, higher values occur near $(1,1)$, lower near $(0,0)$.

\textbf{Applications:}
Modeling real-world phenomena with interdependent variables.

\section*{Conclusion}

In this lecture, we:
\begin{itemize}
  \item Defined joint densities and their properties.
  \item Explored uniform and non-uniform examples.
  \item Highlighted applications in data science and probability.
\end{itemize}

Jointly continuous random variables provide powerful tools for modeling and analyzing multidimensional continuous data.

\end{document}

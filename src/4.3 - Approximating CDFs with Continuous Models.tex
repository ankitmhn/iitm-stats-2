\documentclass{article}
\usepackage{amsmath}
\usepackage{amssymb}
\usepackage{geometry}

\geometry{margin=1in}

\title{Lecture Summary: Approximating CDFs with Continuous Models}
\author{}
\date{}

\begin{document}

\maketitle

\section*{Source: Lecture 4.3.docx}

\section*{Key Points}

\begin{itemize}
  \item \textbf{Transitioning from Discrete to Continuous CDFs:}
    \begin{itemize}
      \item As the alphabet size of a discrete random variable increases, the stepwise nature of its cumulative distribution function (CDF) begins to resemble a continuous line.
      \item Continuous CDFs provide a simpler description, avoiding the complexity of stepwise structures in large datasets.
    \end{itemize}

  \item \textbf{Advantages of Continuous CDFs:}
    \begin{itemize}
      \item Simplifies probability calculations, especially for large datasets.
      \item Shifts focus from exact values to intervals, aligning with real-world scenarios.
      \item Reduces computational effort compared to summing probabilities for discrete steps.
    \end{itemize}

  \item \textbf{Examples:}
    \begin{itemize}
      \item \textbf{Binomial Distribution:}
        \begin{itemize}
          \item For $n = 10$, steps in the CDF are visible, with jumps at each integer.
          \item For $n = 100$, the steps shrink, and the CDF appears smoother.
          \item For large $n$, the CDF becomes indistinguishable from a continuous curve, allowing for approximation.
        \end{itemize}
      \item \textbf{Meteorite Weights:}
        \begin{itemize}
          \item Raw data spans from $0.01$ grams to $60$ tons.
          \item Histogram-derived CDFs show small steps that approximate a smooth curve.
          \item Continuous approximations capture essential probabilities without requiring precise stepwise values.
        \end{itemize}
    \end{itemize}

  \item \textbf{Continuous Approximation of Uniform Distributions:}
    \begin{itemize}
      \item For $X \sim \text{Uniform}\{1, 2, \dots, 100\}$:
        \begin{itemize}
          \item Discrete CDF: Stepwise increments of $1/100$.
          \item Continuous approximation: Linear function $F_X(x) = \frac{x}{100}$ for $x \in [1, 100]$.
        \end{itemize}
      \item Approximation simplifies interval probability calculations:
        \[
          P(3.2 \leq X \leq 10.6) \approx F_X(10.6) - F_X(3.2) = \frac{10.6}{100} - \frac{3.2}{100} = 0.074.
        \]
      \item Exact CDF yields $0.07$, showing a small difference due to approximation.
    \end{itemize}

  \item \textbf{Continuous Approximations for Binomial Distributions:}
    \begin{itemize}
      \item Discrete binomial CDFs are cumbersome for large $n$.
      \item Simple continuous approximations, like logistic functions, provide smooth alternatives:
        \[
          F_X(x) \approx \frac{1}{1 + e^{-k(x - \mu)}},
        \]
        where $\mu$ is the mean, and $k$ is a scaling parameter.
      \item These approximations are accurate for calculating probabilities over intervals, with slight deviations.
    \end{itemize}

  \item \textbf{Theory of Continuous CDFs:}
    \begin{itemize}
      \item A function $F(x)$ is a valid CDF if:
        \begin{itemize}
          \item $F(x)$ is non-decreasing.
          \item $\lim_{x \to -\infty} F(x) = 0$ and $\lim_{x \to \infty} F(x) = 1$.
          \item $F(x)$ is right-continuous.
        \end{itemize}
      \item Continuous CDFs provide a framework for transitioning from discrete to continuous random variables.
    \end{itemize}
\end{itemize}

\section*{Simplified Explanation}

\textbf{Why Approximate CDFs?}
As datasets grow, discrete CDFs with many steps become impractical. Continuous CDFs offer simpler and computationally efficient alternatives.

\textbf{Examples:}
- \textbf{Uniform Distribution:} Replace stepwise increments with a linear approximation.
- \textbf{Binomial Distribution:} Use smooth functions to approximate large-step CDFs.

\textbf{Key Concept:}
Continuous CDFs are non-decreasing functions starting at 0 and ending at 1. They simplify probability calculations over intervals.

\section*{Conclusion}

In this lecture, we:
\begin{itemize}
  \item Transitioned from discrete to continuous CDFs for large datasets.
  \item Discussed the benefits of approximating discrete CDFs with continuous functions.
  \item Highlighted theoretical properties of continuous CDFs.
\end{itemize}

Continuous approximations enhance efficiency and clarity in statistical modeling and probability calculations.

\end{document}

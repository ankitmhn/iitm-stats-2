\documentclass{article}
\usepackage{amsmath}
\usepackage{amssymb}
\usepackage{geometry}
\usepackage{graphicx}

\geometry{margin=1in}

\title{Lecture Summary: Illustrations with Data}
\author{}
\date{}

\begin{document}

\maketitle

\section*{Source: lec7.3.pdf}

\section*{Key Points}

\begin{itemize}
  \item \textbf{Purpose:}
    \begin{itemize}
      \item Explore real-world datasets to compute sample statistics and infer insights.
      \item Assess the applicability of iid sampling models for such datasets.
    \end{itemize}

  \item \textbf{Iris Dataset:}
    \begin{itemize}
      \item Data:
        \begin{itemize}
          \item Three classes (0, 1, 2) with 50 samples each.
          \item Features: Sepal length, Sepal width, Petal length, Petal width.
        \end{itemize}
      \item Analysis:
        \begin{itemize}
          \item Compute sample mean, variance, and proportions for features like sepal length.
          \item Example:
            \[
              S(\text{sepal length} > 5) = \frac{22}{50}, \quad S(4.8 \leq \text{sepal length} \leq 5.2) = \frac{20}{50}.
            \]
        \end{itemize}
      \item Observations:
        \begin{itemize}
          \item Model the data as iid samples from an unknown distribution.
          \item While the iid model seems reasonable for this dataset, it's a first-order approximation.
        \end{itemize}
    \end{itemize}

  \item \textbf{Taj Mahal Air Quality Dataset:}
    \begin{itemize}
      \item Data:
        \begin{itemize}
          \item 11 observations of pollution levels (SO2, NO2, PM2.5, PM10) in April 2021.
          \item Maximum allowable limits: 80 (SO2, NO2), 60 (PM2.5), 100 (PM10).
        \end{itemize}
      \item Statistics:
        \begin{itemize}
          \item Sample means and variances computed for all pollutants.
          \item Proportions:
            \[
              P(\text{exceeds max}) =
              \begin{cases}
                0, & \text{SO2, NO2}, \\
                \frac{7}{11}, & \text{PM2.5}, \\
                \frac{11}{11}, & \text{PM10}.
              \end{cases}
            \]
        \end{itemize}
      \item Observations:
        \begin{itemize}
          \item The iid sampling model may not be appropriate due to temporal correlations.
          \item External factors (e.g., fires, seasonal effects) could influence the data.
        \end{itemize}
      \item Limitations:
        \begin{itemize}
          \item Small dataset with 11 observations is insufficient for strong statistical conclusions.
        \end{itemize}
    \end{itemize}

  \item \textbf{IPL Dataset:}
    \begin{itemize}
      \item Data:
        \begin{itemize}
          \item Runs scored on first three deliveries of IPL matches (1598 innings).
        \end{itemize}
      \item Statistics:
        \begin{itemize}
          \item Sample means and variances:
            \[
              \bar{X}_{0.1} = 0.73, \quad \bar{X}_{0.2} = 0.87, \quad \bar{X}_{0.3} = 0.95.
            \]
          \item Proportions:
            \[
              P(\text{dot ball}) =
              \begin{cases}
                0.5989, & 0.1, \\
                0.55, & 0.2, \\
                0.53, & 0.3.
              \end{cases}
            \]
            \[
              P(\text{boundary}) =
              \begin{cases}
                0.1, & 0.1, \\
                0.1145, & 0.2, \\
                0.13, & 0.3.
              \end{cases}
            \]
        \end{itemize}
      \item Observations:
        \begin{itemize}
          \item Clear trends: Runs and boundaries increase from 0.1 to 0.3.
          \item The iid model is reasonable but may require further checks for dependencies (e.g., psychological effects on bowlers after a boundary).
        \end{itemize}
    \end{itemize}

  \item \textbf{Lessons Learned:}
    \begin{itemize}
      \item Statistical models and conclusions depend on the dataset size, quality, and context.
      \item Large datasets (like IPL) allow for more reliable inferences compared to smaller datasets (like Taj Mahal).
    \end{itemize}
\end{itemize}

\section*{Simplified Explanation}

\textbf{Key Datasets Analyzed:}
1. \textbf{Iris Dataset:} Modeled as iid samples; computed basic statistics.
2. \textbf{Taj Mahal Air Quality:} Insufficient data for strong conclusions; temporal effects likely.
3. \textbf{IPL Data:} Large dataset revealing trends; iid model reasonable with caveats.

\textbf{Insights:}
- Larger datasets provide stronger confidence in statistical stories.
- Sample statistics offer valuable summaries but must be interpreted in context.

\section*{Conclusion}

In this lecture, we:
\begin{itemize}
  \item Explored sample statistics through three datasets.
  \item Discussed the suitability of iid models for different contexts.
  \item Highlighted the role of data size and context in statistical conclusions.
\end{itemize}

Real-world datasets illustrate the importance of understanding the assumptions and limitations behind statistical models.

\end{document}

\documentclass{article}
\usepackage{amsmath}
\usepackage{amssymb}
\usepackage{geometry}

\geometry{margin=1in}

\title{Lecture Summary: Conditioning with Multiple Discrete Random Variables}
\author{}
\date{}

\begin{document}

\maketitle

\section*{Source: Lec1.7.pdf}

\section*{Key Points}

\begin{itemize}
  \item \textbf{Relationship Between Marginals, Conditionals, and Joint PMFs:}
    \begin{itemize}
      \item Marginalization reduces complexity by focusing on smaller subsets of variables.
      \item Conditional PMFs bridge marginals and the joint PMF, enabling detailed analysis.
      \item Key formula:
        \[
          f_{X_1, X_2, \dots, X_n}(t_1, t_2, \dots, t_n) = f_{X_1}(t_1) \prod_{i=2}^n f_{X_i \mid X_1, \dots, X_{i-1}}(t_i \mid t_1, \dots, t_{i-1}).
        \]
    \end{itemize}

  \item \textbf{Basic Conditioning:}
    \begin{itemize}
      \item For two variables $X_1$ and $X_2$:
        \[
          f_{X_1 \mid X_2}(t_1 \mid t_2) = \frac{f_{X_1, X_2}(t_1, t_2)}{f_{X_2}(t_2)}.
        \]
      \item This formula generalizes to multiple variables:
        \[
          f_{X_1, X_2 \mid X_3}(t_1, t_2 \mid t_3) = \frac{f_{X_1, X_2, X_3}(t_1, t_2, t_3)}{f_{X_3}(t_3)}.
        \]
    \end{itemize}

  \item \textbf{Examples of Conditioning:}
    \begin{itemize}
      \item \textbf{Example 1: Pairwise Conditioning:}
        \begin{itemize}
          \item $X_1 \mid X_2 = t_2$: Conditional PMF is derived as:
            \[
              f_{X_1 \mid X_2}(t_1 \mid t_2) = \frac{f_{X_1, X_2}(t_1, t_2)}{f_{X_2}(t_2)}.
            \]
        \end{itemize}
      \item \textbf{Example 2: Conditional on Multiple Variables:}
        \begin{itemize}
          \item For $X_1 \mid (X_2 = t_2, X_3 = t_3)$:
            \[
              f_{X_1 \mid X_2, X_3}(t_1 \mid t_2, t_3) = \frac{f_{X_1, X_2, X_3}(t_1, t_2, t_3)}{f_{X_2, X_3}(t_2, t_3)}.
            \]
        \end{itemize}
      \item \textbf{Example 3: Joint Conditional PMF:}
        \[
          f_{X_1, X_3 \mid X_2, X_4}(t_1, t_3 \mid t_2, t_4) = \frac{f_{X_1, X_2, X_3, X_4}(t_1, t_2, t_3, t_4)}{f_{X_2, X_4}(t_2, t_4)}.
        \]
    \end{itemize}

  \item \textbf{Factoring the Joint PMF:}
    \begin{itemize}
      \item Any joint PMF can be factored iteratively:
        \[
          f_{X_1, X_2, \dots, X_n}(t_1, t_2, \dots, t_n) = f_{X_n}(t_n) \prod_{i=1}^{n-1} f_{X_i \mid X_{i+1}, \dots, X_n}(t_i \mid t_{i+1}, \dots, t_n).
        \]
      \item This flexibility allows reordering variables, e.g., $f_{X_3 \mid X_1, X_2}$ or $f_{X_1 \mid X_2, X_3}$.
    \end{itemize}

  \item \textbf{Applications:}
    \begin{itemize}
      \item Conditioning simplifies computations in high-dimensional problems.
      \item Factorization helps analyze dependencies and decompose complex distributions.
      \item Practical examples include evaluating conditional probabilities in tabular data.
    \end{itemize}
\end{itemize}

\section*{Simplified Explanation}

\textbf{Conditioning and Marginalization:}
Conditioning relates individual and joint distributions, while marginalization reduces variables.

\textbf{Factoring:}
Break down a joint PMF into products of marginals and conditionals, simplifying analysis.

\textbf{Example:}
For $f_{X_1, X_2, X_3}(t_1, t_2, t_3)$:
\[
  f_{X_1, X_2, X_3}(t_1, t_2, t_3) = f_{X_1}(t_1) \cdot f_{X_2 \mid X_1}(t_2 \mid t_1) \cdot f_{X_3 \mid X_1, X_2}(t_3 \mid t_1, t_2).
\]

\section*{Conclusion}

In this lecture, we:
\begin{itemize}
  \item Defined conditional PMFs for multiple random variables.
  \item Demonstrated the utility of factoring joint PMFs into simpler components.
  \item Highlighted the flexibility of variable ordering in factorization.
\end{itemize}

These principles are foundational for probabilistic analysis in high-dimensional spaces.

\end{document}

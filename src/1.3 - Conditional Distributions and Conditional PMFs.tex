\documentclass{article}
\usepackage{amsmath}
\usepackage{amssymb}
\usepackage{geometry}

\geometry{margin=1in}

\title{Lecture Summary: Conditional Distributions and Conditional PMFs}
\author{}
\date{}

\begin{document}

\maketitle

\section*{Source: Lec1.3.pdf}

\section*{Key Points}

\begin{itemize}
  \item \textbf{Introduction to Conditional PMFs:}
    \begin{itemize}
      \item The conditional PMF of a random variable $X$ given an event $A$ is defined as:
        \[
          P(X = t \mid A) = \frac{P(X = t \cap A)}{P(A)}.
        \]
      \item Conditioning can change the range of the random variable. For example, values of $X$ outside $A$ may no longer be possible.
    \end{itemize}

  \item \textbf{Extending to Two Random Variables:}
    \begin{itemize}
      \item For two random variables $X$ and $Y$ with joint PMF $f_{XY}(t_1, t_2)$:
        \[
          f_{Y \mid X}(t_2 \mid t_1) = \frac{f_{XY}(t_1, t_2)}{f_X(t_1)}.
        \]
      \item The numerator is the joint PMF $f_{XY}$, and the denominator is the marginal PMF $f_X$.
    \end{itemize}

  \item \textbf{Properties of Conditional PMFs:}
    \begin{itemize}
      \item The conditional PMF is a valid PMF, meaning:
        \[
          \sum_{t_2} f_{Y \mid X}(t_2 \mid t_1) = 1.
        \]
      \item Conditional PMFs allow calculation of joint PMFs using:
        \[
          f_{XY}(t_1, t_2) = f_{Y \mid X}(t_2 \mid t_1) \cdot f_X(t_1).
        \]
    \end{itemize}

  \item \textbf{Examples:}
    \begin{itemize}
      \item \textbf{Example 1: Joint PMF of $X$ and $Y$:}
        \begin{itemize}
          \item Joint PMF is given as a table.
          \item Marginal PMFs $f_X$ and $f_Y$ are computed by summing rows and columns, respectively.
          \item Conditional PMF $f_{Y \mid X}$ is computed by dividing joint probabilities by the marginal $f_X$.
        \end{itemize}

      \item \textbf{Example 2: Lottery Numbers:}
        \begin{itemize}
          \item Joint PMF of $X$ (units digit) and $Y$ (remainder modulo 4).
          \item Conditional PMF $f_{Y \mid X}$ reflects how $Y$ depends on $X$.
        \end{itemize}

      \item \textbf{Example 3: General Calculation:}
        \begin{itemize}
          \item From a joint PMF table, compute $f_{Y \mid X}(t_2 \mid t_1)$ for specific $t_1$ by normalizing the corresponding row or column.
        \end{itemize}
    \end{itemize}

  \item \textbf{Useful Identities:}
    \begin{itemize}
      \item The relationship between joint, marginal, and conditional PMFs is:
        \[
          f_{XY}(t_1, t_2) = f_{Y \mid X}(t_2 \mid t_1) \cdot f_X(t_1) = f_{X \mid Y}(t_1 \mid t_2) \cdot f_Y(t_2).
        \]
      \item Summing conditional PMFs over the range of the conditioned variable gives 1:
        \[
          \sum_{t_2} f_{Y \mid X}(t_2 \mid t_1) = 1.
        \]
    \end{itemize}
\end{itemize}

\section*{Simplified Explanation}

\textbf{Conditional PMFs:}
Probability distributions of one random variable given specific information about another variable or event.

\textbf{Example:}
If $f_{XY}(0, 1) = 1/8$ and $f_X(0) = 3/8$, then:
\[
  f_{Y \mid X}(1 \mid 0) = \frac{f_{XY}(0, 1)}{f_X(0)} = \frac{1/8}{3/8} = \frac{1}{3}.
\]

\textbf{Key Formula:}
The joint PMF can be reconstructed from the conditional and marginal PMFs:
\[
  f_{XY}(t_1, t_2) = f_{Y \mid X}(t_2 \mid t_1) \cdot f_X(t_1).
\]

\section*{Conclusion}

In this lecture, we:
\begin{itemize}
  \item Defined conditional PMFs for single and multiple random variables.
  \item Explored examples to compute conditional probabilities.
  \item Highlighted the relationships between joint, marginal, and conditional PMFs.
\end{itemize}

Conditional PMFs are crucial for understanding dependencies between random variables and solving complex probabilistic problems.

\end{document}

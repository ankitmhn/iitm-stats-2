\documentclass{article}
\usepackage{amsmath}
\usepackage{amssymb}
\usepackage{geometry}

\geometry{margin=1in}

\title{Lecture Summary: Conditional Densities}
\author{}
\date{}

\begin{document}

\maketitle

\section*{Source: Lecture 5.6.docx}

\section*{Key Points}

\begin{itemize}
  \item \textbf{Definition of Conditional Density:}
    \begin{itemize}
      \item The conditional density of $Y$ given $X = a$ is defined as:
        \[
          f_{Y|X}(y|a) = \frac{f_{X,Y}(a, y)}{f_X(a)}, \quad \text{for } f_X(a) > 0.
        \]
      \item This density describes the distribution of $Y$ for a fixed value of $X = a$.
      \item Even though $P(X = a) = 0$ for continuous variables, the density is valid and derived as a limiting concept.
    \end{itemize}

  \item \textbf{Properties of Conditional Density:}
    \begin{itemize}
      \item $f_{Y|X}(y|a) \geq 0$ for all $y$.
      \item The total integral over $y$ equals 1:
        \[
          \int_{-\infty}^\infty f_{Y|X}(y|a) \, dy = 1.
        \]
      \item The joint density can be expressed in terms of the conditional density:
        \[
          f_{X,Y}(x,y) = f_X(x) f_{Y|X}(y|x).
        \]
    \end{itemize}

  \item \textbf{Examples:}
    \begin{itemize}
      \item \textbf{Uniform Distribution on $[0, 1] \times [0, 1]$:}
        \begin{itemize}
          \item Joint density: $f_{X,Y}(x,y) = 1$ for $0 \leq x, y \leq 1$.
          \item Marginals: $f_X(x) = 1$ and $f_Y(y) = 1$.
          \item Conditional densities:
            \[
              f_{Y|X}(y|x) = 1, \quad f_{X|Y}(x|y) = 1, \quad \text{for } 0 \leq x, y \leq 1.
            \]
        \end{itemize}

      \item \textbf{Non-Uniform Example:}
        \begin{itemize}
          \item Joint density: $f_{X,Y}(x,y) = x + y$ for $0 \leq x, y \leq 1$.
          \item Marginals:
            \[
              f_X(x) = \int_0^1 (x + y) \, dy = x + \frac{1}{2}, \quad f_Y(y) = \int_0^1 (x + y) \, dx = y + \frac{1}{2}.
            \]
          \item Conditional density of $Y|X = a$:
            \[
              f_{Y|X}(y|a) = \frac{a + y}{a + \frac{1}{2}}, \quad 0 \leq y \leq 1.
            \]
        \end{itemize}
    \end{itemize}

  \item \textbf{Visualizing Conditional Densities:}
    \begin{itemize}
      \item Conditional densities can be viewed as slices of the joint density at specific values of $X$ or $Y$.
      \item For example, fixing $X = a$ gives the distribution of $Y$ conditional on $X = a$.
    \end{itemize}

  \item \textbf{Applications:}
    \begin{itemize}
      \item Conditional densities are essential for understanding relationships between variables.
      \item Useful in regression, where $Y$ is modeled given $X$.
      \item Enables computation of probabilities in constrained scenarios.
    \end{itemize}
\end{itemize}

\section*{Simplified Explanation}

\textbf{What is a Conditional Density?}
- Describes how one variable (e.g., $Y$) behaves given a specific value of another variable (e.g., $X = a$).
- Formula:
\[
  f_{Y|X}(y|a) = \frac{f_{X,Y}(a, y)}{f_X(a)}.
\]

\textbf{Example:}
For a non-uniform joint density $f_{X,Y}(x,y) = x + y$:
\[
  f_{Y|X}(y|a) = \frac{a + y}{a + \frac{1}{2}}, \quad 0 \leq y \leq 1.
\]

\textbf{Key Idea:}
Conditional densities are derived by normalizing slices of the joint density.

\section*{Conclusion}

In this lecture, we:
\begin{itemize}
  \item Defined conditional densities for continuous random variables.
  \item Explored examples and derived properties.
  \item Highlighted practical applications in probability and statistics.
\end{itemize}

Understanding conditional densities is crucial for analyzing relationships and dependencies in continuous random variables.

\end{document}

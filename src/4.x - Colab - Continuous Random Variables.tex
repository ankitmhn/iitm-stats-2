\documentclass{article}
\usepackage{amsmath}
\usepackage{amssymb}
\usepackage{geometry}
\usepackage{graphicx}

\geometry{margin=1in}

\title{Lecture Summary: Continuous Random Variables - Colab Simulation}
\author{}
\date{}

\begin{document}

\maketitle

\section*{Source: Lecture 4.9.docx}

\section*{Key Points}

\begin{itemize}
  \item \textbf{Purpose:}
    \begin{itemize}
      \item Simulate continuous random variables in Python using Google Colab.
      \item Explore Monte Carlo techniques for estimating probabilities and comparing theoretical and empirical results.
    \end{itemize}

  \item \textbf{Packages Used:}
    \begin{itemize}
      \item \texttt{NumPy}: For generating numerical data.
      \item \texttt{scipy.stats}: For generating random samples from continuous distributions.
      \item \texttt{matplotlib.pyplot}: For plotting histograms and densities.
    \end{itemize}

  \item \textbf{Key Steps:}
    \begin{enumerate}
      \item Import necessary libraries:
        \begin{verbatim}
        import numpy as np
        from scipy.stats import uniform, expon, norm
        import matplotlib.pyplot as plt
        \end{verbatim}

      \item Generate samples from continuous distributions using \texttt{scipy.stats}:
        \begin{itemize}
          \item Uniform distribution: \texttt{st.uniform.rvs(loc, scale, size)}.
          \item Exponential distribution: \texttt{st.expon.rvs(scale, size)}.
          \item Normal distribution: \texttt{st.norm.rvs(loc, scale, size)}.
        \end{itemize}

      \item Plot histograms and overlay theoretical PDFs for comparison.
    \end{enumerate}

  \item \textbf{Examples and Results:}
    \begin{itemize}
      \item \textbf{Uniform Distribution:}
        \begin{itemize}
          \item Samples generated in range $[0, 3]$ with 10,000 values.
          \item Histogram with bins overlaid by theoretical density $f_X(x) = \frac{1}{3}$ for $0 \leq x \leq 3$.
          \item Visual comparison confirms empirical data approximates the theoretical PDF.
        \end{itemize}

      \item \textbf{Exponential Distribution:}
        \begin{itemize}
          \item Parameter $\lambda = 2$, 10,000 samples generated.
          \item Histogram plotted with theoretical density $f_X(x) = 2e^{-2x}$.
          \item Density decays rapidly, matching the histogram.
        \end{itemize}

      \item \textbf{Normal Distribution:}
        \begin{itemize}
          \item Mean $\mu = 0$, standard deviation $\sigma = 1$.
          \item Histogram of samples overlaid with theoretical PDF.
          \item Matches closely, showing the bell-shaped curve.
        \end{itemize}
    \end{itemize}

  \item \textbf{Practical Insights:}
    \begin{itemize}
      \item Histograms provide a visual representation of data distribution and are useful for verifying the fit of a theoretical model.
      \item Comparing histograms with theoretical PDFs helps identify suitable distributions for data modeling.
      \item Adjusting bin sizes in histograms can improve visualization and interpretation.
    \end{itemize}

  \item \textbf{Monte Carlo Simulations:}
    \begin{itemize}
      \item Generate random samples to estimate probabilities and distribution properties.
      \item Verify theoretical results through empirical data.
      \item Useful for exploring large datasets where analytical solutions are complex.
    \end{itemize}
\end{itemize}

\section*{Simplified Explanation}

\textbf{Objective:} Simulate and analyze continuous random variables using Python.

\textbf{Examples:}
1. Uniform: Constant density across an interval.
2. Exponential: Rapidly decaying density.
3. Normal: Symmetric bell curve.

\textbf{Tools:}
- Histograms to visualize data.
- PDFs to verify theoretical distributions.

\textbf{Key Takeaway:} Simulation bridges theory and practice, offering insights into real-world data distributions.

\section*{Conclusion}

In this lecture, we:
\begin{itemize}
  \item Explored continuous random variable simulation in Colab.
  \item Visualized data using histograms and theoretical densities.
  \item Discussed the use of Monte Carlo techniques for empirical validation.
\end{itemize}

Simulation is a powerful tool for understanding and modeling continuous random variables, providing practical verification of theoretical concepts.

\end{document}

\documentclass{article}
\usepackage{amsmath}
\usepackage{amssymb}
\usepackage{geometry}

\geometry{margin=1in}

\title{Lecture Summary: Distribution of the Sum of Two Random Variables}
\author{}
\date{}

\begin{document}

\maketitle

\section*{Source: Lecture 2.8.pdf}

\section*{Key Points}

\begin{itemize}
  \item \textbf{Overview:}
    \begin{itemize}
      \item The focus is on deriving the probability mass function (PMF) of a function of two random variables, particularly their sum.
      \item Process involves two key steps:
        \begin{enumerate}
          \item Determine the range of possible values for $Z = X + Y$.
          \item Sum over the contours defined by $g(x, y) = Z$ in the joint PMF.
        \end{enumerate}
    \end{itemize}

  \item \textbf{Steps to Derive the PMF of $Z$:}
    \begin{enumerate}
      \item \textbf{Find the Range of $Z$:}
        \begin{itemize}
          \item Identify all possible values $Z$ can take based on the ranges of $X$ and $Y$.
          \item Example: If $X, Y \in \{1, 2, 3, 4, 5, 6\}$ (as in a dice roll), then $Z = X + Y$ can range from $2$ to $12$.
        \end{itemize}
      \item \textbf{Sum Over Contours:}
        \begin{itemize}
          \item For each possible value of $Z$, identify all pairs $(x, y)$ such that $x + y = Z$.
          \item Add the probabilities of these pairs using the joint PMF.
          \item Visualization can aid this process, either graphically or by pattern recognition.
        \end{itemize}
    \end{enumerate}

  \item \textbf{Example: Sum of Two Dice Rolls:}
    \begin{itemize}
      \item \textbf{Step 1: Range:}
        \[
          Z = X + Y \in \{2, 3, \dots, 12\}.
        \]
      \item \textbf{Step 2: Contours:}
        \begin{itemize}
          \item $Z = 2$: $(1, 1)$.
          \item $Z = 3$: $(1, 2), (2, 1)$.
          \item $Z = 4$: $(1, 3), (2, 2), (3, 1)$.
          \item Continue for all values up to $Z = 12$.
        \end{itemize}
      \item Compute PMF:
        \[
          P(Z = z) = \frac{\text{Number of pairs } (x, y) \text{ such that } x + y = z}{36}.
        \]
      \item Example:
        \[
          P(Z = 2) = \frac{1}{36}, \quad P(Z = 3) = \frac{2}{36}, \quad P(Z = 4) = \frac{3}{36}.
        \]
      \item Symmetry: $P(Z = z)$ increases to a peak at $Z = 7$ and decreases symmetrically.
    \end{itemize}

  \item \textbf{Visualization:}
    \begin{itemize}
      \item Graphical representation involves plotting $x$ and $y$ on a grid and identifying points where $x + y = Z$.
      \item Contours corresponding to $x + y = Z$ are diagonal lines, each containing points that contribute to $P(Z = z)$.
    \end{itemize}

  \item \textbf{Applications:}
    \begin{itemize}
      \item This approach generalizes to sums of variables with other distributions or continuous cases.
      \item The method emphasizes systematic counting and pattern recognition.
    \end{itemize}
\end{itemize}

\section*{Simplified Explanation}

\textbf{Key Idea:}
The PMF of $Z = X + Y$ is derived by summing probabilities of all pairs $(x, y)$ such that $x + y = Z$.

\textbf{Example:}
For two dice:
- $Z = 2$: One pair $(1, 1)$, so $P(Z = 2) = \frac{1}{36}$.
- $Z = 3$: Two pairs $(1, 2), (2, 1)$, so $P(Z = 3) = \frac{2}{36}$.

\textbf{Visualization:}
Contours represent combinations of $(x, y)$ with the same $Z$. The PMF is computed by counting points on each contour.

\section*{Conclusion}

In this lecture, we:
\begin{itemize}
  \item Discussed how to derive the PMF of the sum of two random variables.
  \item Used examples to illustrate the process of determining the range and summing over contours.
  \item Emphasized the role of visualization in simplifying the computation.
\end{itemize}

This method is foundational for understanding distributions of sums and other functions of random variables.

\end{document}

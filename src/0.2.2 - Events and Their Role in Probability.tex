\documentclass{article}
\usepackage{amsmath}
\usepackage{amssymb}
\usepackage{geometry}

\geometry{margin=1in}

\title{Lecture Summary: Events and Their Role in Probability Theory}
\author{}
\date{}

\begin{document}

\maketitle

\section*{Source: Lecture 02.2.pdf}

\section*{Key Points}

\begin{itemize}
  \item \textbf{Definition of an Event:}
    \begin{itemize}
      \item An event is defined as a subset of the sample space $S$, which represents all possible outcomes of an experiment.
      \item While this definition is widely used, certain advanced cases may impose restrictions on which subsets qualify as events. These technicalities are typically ignored in practical problem-solving.
      \item Analogy: Similar to how Newtonian mechanics works well for most cases but requires corrections in extreme conditions (e.g., relativity), the basic event definition is sufficient for most practical applications.
    \end{itemize}

  \item \textbf{Examples of Events:}
    \begin{itemize}
      \item \textbf{Coin Toss:}
        \begin{itemize}
          \item Sample Space: $S = \{\text{Heads}, \text{Tails}\}$.
          \item Possible Events:
            \begin{itemize}
              \item $\emptyset$ (no outcome occurred).
              \item $\{\text{Heads}\}$.
              \item $\{\text{Tails}\}$.
              \item $\{\text{Heads}, \text{Tails}\}$ (entire sample space).
            \end{itemize}
        \end{itemize}

      \item \textbf{Die Roll:}
        \begin{itemize}
          \item Sample Space: $S = \{1, 2, 3, 4, 5, 6\}$.
          \item Possible Events:
            \begin{itemize}
              \item Single outcomes: $\{1\}, \{2\}, \dots, \{6\}$.
              \item Combinations of outcomes: $\{2, 4, 6\}$ (even numbers), $\{3, 6\}$ (multiples of 3), etc.
              \item Total subsets: $2^6 = 64$.
            \end{itemize}
        \end{itemize}

      \item \textbf{Fishing Example:}
        \begin{itemize}
          \item Experiment: A fisherman catches fish.
          \item Sample Space: All possible catches (e.g., types of fish, quantities).
          \item Example Events:
            \begin{itemize}
              \item Catching more than 100 kg of fish.
              \item Catching specific types of fish (e.g., Seer fish).
              \item Events may not require explicit enumeration of the entire sample space.
            \end{itemize}
        \end{itemize}
    \end{itemize}

  \item \textbf{Interpreting Events:}
    \begin{itemize}
      \item An event occurs if the actual outcome of an experiment belongs to the event subset.
      \item Example: In a die roll, the event $\{2, 4, 6\}$ occurs if the outcome is 2, 4, or 6.
    \end{itemize}

  \item \textbf{Set Operations on Events:}
    \begin{itemize}
      \item Events are subsets of the sample space, and all set operations apply:
        \begin{itemize}
          \item \textbf{Union:} The event $A \cup B$ occurs if either $A$ or $B$ (or both) occur.
          \item \textbf{Intersection:} The event $A \cap B$ occurs if both $A$ and $B$ occur.
          \item \textbf{Complement:} The event $A^c$ occurs if $A$ does not occur.
          \item \textbf{Containment:} $A \subseteq B$ if every outcome in $A$ is also in $B$.
        \end{itemize}
      \item These operations often have physical interpretations in terms of experiments.
    \end{itemize}
\end{itemize}

\section*{Simplified Explanation}

\textbf{What Is an Event?}
An event is a subset of possible outcomes from an experiment, and it represents situations of interest.

\textbf{Examples:}
- Coin toss: Event $\{\text{Heads}\}$ occurs if the coin lands on heads.
- Die roll: Event $\{2, 4, 6\}$ (even numbers) occurs if the outcome is 2, 4, or 6.

\textbf{Set Operations:}
Events are sets, so operations like union, intersection, and complement apply, with meaningful interpretations in probability.

\section*{Conclusion}

In this lecture, we:
\begin{itemize}
  \item Defined events and explained their role in probability theory.
  \item Used examples to clarify the concept of events and their occurrence.
  \item Highlighted the importance of set operations in understanding events.
\end{itemize}

Events are fundamental to probability theory and essential for studying the likelihood of outcomes.

\end{document}

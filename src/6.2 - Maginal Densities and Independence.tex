\documentclass{article}
\usepackage{amsmath}
\usepackage{amssymb}
\usepackage{geometry}
\usepackage{graphicx}

\geometry{margin=1in}

\title{Lecture Summary: Marginal Densities and Independence}
\author{}
\date{}

\begin{document}

\maketitle

\section*{Source: Lecture 5.4.docx}

\section*{Key Points}

\begin{itemize}
  \item \textbf{Marginal Densities:}
    \begin{itemize}
      \item For two jointly continuous random variables $X$ and $Y$ with joint density $f_{X,Y}(x,y)$:
        \[
          f_X(x) = \int_{-\infty}^\infty f_{X,Y}(x,y) \, dy, \quad f_Y(y) = \int_{-\infty}^\infty f_{X,Y}(x,y) \, dx.
        \]
      \item Marginal densities describe individual distributions of $X$ and $Y$, integrating out the other variable.
      \item Visualization: Imagine a 3D density function where slicing along one axis yields a 2D marginal density function.
    \end{itemize}

  \item \textbf{Independence:}
    \begin{itemize}
      \item $X$ and $Y$ are independent if and only if:
        \[
          f_{X,Y}(x,y) = f_X(x)f_Y(y).
        \]
      \item Independence implies that the joint density is the product of the marginals, with no dependency between variables.
    \end{itemize}

  \item \textbf{Uniform Example:}
    \begin{itemize}
      \item Joint density:
        \[
          f_{X,Y}(x,y) =
          \begin{cases}
            1, & 0 \leq x, y \leq 1, \\
            0, & \text{otherwise}.
          \end{cases}
        \]
      \item Marginal densities:
        \[
          f_X(x) = \int_0^1 1 \, dy = 1, \quad f_Y(y) = \int_0^1 1 \, dx = 1.
        \]
      \item Both $f_X(x)$ and $f_Y(y)$ are uniform on $[0, 1]$.
      \item Marginal densities alone do not uniquely determine the joint density.
    \end{itemize}

  \item \textbf{Non-Uniform Example:}
    \begin{itemize}
      \item Joint density:
        \[
          f_{X,Y}(x,y) =
          \begin{cases}
            2, & 0 \leq x \leq y \leq 1, \\
            0, & \text{otherwise}.
          \end{cases}
        \]
      \item Marginal densities:
        \[
          f_X(x) = \int_x^1 2 \, dy = 2(1-x), \quad f_Y(y) = \int_0^y 2 \, dx = 2y.
        \]
      \item This example illustrates non-uniform marginals and non-rectangular support regions.
    \end{itemize}

  \item \textbf{Triangular Support Example:}
    \begin{itemize}
      \item Joint density:
        \[
          f_{X,Y}(x,y) =
          \begin{cases}
            1, & 0 \leq x \leq y \leq 1, \\
            0, & \text{otherwise}.
          \end{cases}
        \]
      \item Marginal densities:
        \[
          f_X(x) = \int_x^1 1 \, dy = 1-x, \quad f_Y(y) = \int_0^y 1 \, dx = y.
        \]
      \item Triangular regions illustrate how support impacts marginal densities.
    \end{itemize}

  \item \textbf{Key Insights:}
    \begin{itemize}
      \item Marginal densities can be calculated from joint densities, but the reverse is not unique.
      \item Joint densities must be carefully modeled as different joint densities can yield the same marginals.
    \end{itemize}
\end{itemize}

\section*{Simplified Explanation}

\textbf{Marginal Densities:}
Describe the individual distributions of $X$ and $Y$ by integrating the joint density.

\textbf{Independence:}
$X$ and $Y$ are independent if their joint density equals the product of their marginals.

\textbf{Example:}
1. Uniform on $[0,1]^2$: Marginals are uniform, but marginals alone don't define the joint density.
2. Non-uniform regions: Triangular support changes marginals.

\textbf{Key Idea:}
Marginals describe individual variables, but the joint density shows how they are connected.

\section*{Conclusion}

In this lecture, we:
\begin{itemize}
  \item Defined marginal densities and their computation.
  \item Discussed independence for continuous random variables.
  \item Highlighted examples showing how joint densities impact marginals.
\end{itemize}

Understanding marginal densities and independence is essential for analyzing relationships between continuous random variables.

\end{document}

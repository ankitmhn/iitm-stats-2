\documentclass{article}
\usepackage{amsmath}
\usepackage{amssymb}
\usepackage{geometry}

\geometry{margin=1in}

\title{Lecture Summary: Functions of a Continuous Random Variable}
\author{}
\date{}

\begin{document}

\maketitle

\section*{Source: Lecture 4.7.docx}

\section*{Key Points}

\begin{itemize}
  \item \textbf{Introduction:}
    \begin{itemize}
      \item A function of a continuous random variable can generate new random variables by applying transformations.
      \item Common examples include scaling, translation, and nonlinear transformations.
    \end{itemize}

  \item \textbf{Methodology:}
    \begin{itemize}
      \item Start by finding the cumulative distribution function (CDF) of the transformed variable.
      \item If differentiable, derive the probability density function (PDF) from the CDF:
        \[
          f_Y(y) = \frac{d}{dy}F_Y(y).
        \]
    \end{itemize}

  \item \textbf{Scaling Example:}
    \begin{itemize}
      \item If $X \sim \text{Uniform}[0, 1]$ and $Y = 2X$, then:
        \begin{itemize}
          \item $Y \sim \text{Uniform}[0, 2]$ with:
            \[
              f_Y(y) =
              \begin{cases}
                \frac{1}{2}, & 0 \leq y \leq 2, \\
                0, & \text{otherwise}.
              \end{cases}
            \]
        \end{itemize}
    \end{itemize}

  \item \textbf{General Formula for Monotonic Functions:}
    \begin{itemize}
      \item For monotonic functions $g(X)$ with inverse $g^{-1}(y)$, the PDF of $Y = g(X)$ is:
        \[
          f_Y(y) = \frac{f_X(g^{-1}(y))}{\lvert g'(g^{-1}(y)) \rvert}.
        \]
      \item Example:
        \begin{itemize}
          \item If $X \sim \text{Exponential}(\lambda)$ and $Y = X^2$, then:
            \[
              f_Y(y) = \frac{\lambda e^{-\lambda \sqrt{y}}}{2\sqrt{y}}, \quad y > 0.
            \]
        \end{itemize}
    \end{itemize}

  \item \textbf{Non-Monotonic Functions:}
    \begin{itemize}
      \item When $g(X)$ is not monotonic, divide the domain into regions where $g(X)$ is monotonic.
      \item Example:
        \begin{itemize}
          \item If $X \sim \text{Uniform}[-3, 1]$ and $Y = X^2$, the PDF of $Y$ must account for the non-monotonicity:
            \[
              f_Y(y) =
              \begin{cases}
                \frac{1}{2\sqrt{y}}, & 0 \leq y \leq 1, \\
                \frac{1}{4\sqrt{y}}, & 1 < y \leq 9.
              \end{cases}
            \]
        \end{itemize}
    \end{itemize}

  \item \textbf{Special Case: Affine Transformations:}
    \begin{itemize}
      \item If $Y = aX + b$:
        \[
          f_Y(y) = \frac{1}{\lvert a \rvert} f_X\left(\frac{y-b}{a}\right).
        \]
      \item Example:
        \begin{itemize}
          \item $X \sim \text{Normal}(0, 1)$ and $Y = 2X + 3$, then $Y \sim \text{Normal}(3, 4)$.
        \end{itemize}
    \end{itemize}

  \item \textbf{Mixtures and Discontinuities:}
    \begin{itemize}
      \item Functions like $g(X) = \max(X, 0)$ can produce distributions that are mixtures of discrete and continuous parts.
      \item Example:
        \begin{itemize}
          \item If $X \sim \text{Uniform}[-3, 1]$ and $Y = \max(X, 0)$, the resulting distribution is:
            \[
              f_Y(y) =
              \begin{cases}
                \frac{3}{4}, & y = 0, \\
                \frac{1}{4}, & 0 < y \leq 1.
              \end{cases}
            \]
        \end{itemize}
    \end{itemize}

  \item \textbf{General Approach:}
    \begin{itemize}
      \item For any $g(X)$:
        \begin{enumerate}
          \item Identify the regions of monotonicity.
          \item Compute the CDF by integrating the original PDF over appropriate regions.
          \item Derive the PDF if needed.
        \end{enumerate}
    \end{itemize}
\end{itemize}

\section*{Simplified Explanation}

\textbf{Transforming Random Variables:}
Functions of a random variable, such as scaling or squaring, create new distributions.

\textbf{Key Tools:}
- Monotonic functions: Use the formula for PDF transformation.
- Non-monotonic functions: Divide into regions and compute probabilities separately.

\textbf{Examples:}
1. Scaling: $Y = 2X$ scales the range and flattens the PDF.
2. Squaring: $Y = X^2$ creates asymmetry in the PDF.

\textbf{Applications:}
- Transformations allow modeling of derived quantities (e.g., area, volume, or thresholds).

\section*{Conclusion}

In this lecture, we:
\begin{itemize}
  \item Discussed transformations of continuous random variables.
  \item Introduced methods for deriving PDFs for transformed variables.
  \item Highlighted cases with monotonic, non-monotonic, and affine transformations.
\end{itemize}

Understanding transformations is crucial for modeling real-world scenarios involving derived quantities.

\end{document}

\documentclass{article}
\usepackage{amsmath}
\usepackage{amssymb}
\usepackage{geometry}

\geometry{margin=1in}

\title{Lecture Summary: Expected Value}
\author{}
\date{}

\begin{document}

\maketitle

\section*{Source: Lecture 3.1.pdf}

\section*{Key Points}

\begin{itemize}
  \item \textbf{Definition of Expected Value:}
    \begin{itemize}
      \item The expected value (denoted as $E[X]$) is a measure of the "average" value a random variable $X$ takes, weighted by its probabilities.
      \item Formula:
        \[
          E[X] = \sum_{t \in T_X} t \cdot f_X(t),
        \]
        where $T_X$ is the range of $X$ and $f_X(t) = P(X = t)$ is the probability mass function (PMF).
      \item Interpretation: Over repeated trials, the arithmetic mean of $X$ approaches $E[X]$.
    \end{itemize}

  \item \textbf{Importance of Expected Value:}
    \begin{itemize}
      \item Summarizes the central tendency of a random variable.
      \item Provides insights into long-term outcomes (e.g., average gains in a casino game).
      \item Useful in practical decision-making and theoretical analysis.
    \end{itemize}

  \item \textbf{Examples:}
    \begin{itemize}
      \item \textbf{Bernoulli Random Variable:}
        \begin{itemize}
          \item PMF: $P(X = 1) = p$, $P(X = 0) = 1 - p$.
          \item Expected value:
            \[
              E[X] = 0 \cdot (1 - p) + 1 \cdot p = p.
            \]
        \end{itemize}

      \item \textbf{Uniform Random Variable (1 to 6):}
        \begin{itemize}
          \item PMF: $P(X = x) = \frac{1}{6}$ for $x \in \{1, 2, 3, 4, 5, 6\}$.
          \item Expected value:
            \[
              E[X] = \frac{1}{6}(1 + 2 + 3 + 4 + 5 + 6) = 3.5.
            \]
        \end{itemize}

      \item \textbf{Lottery Example:}
        \begin{itemize}
          \item PMF: $P(X = 200) = \frac{1}{1000}$, $P(X = 20) = \frac{27}{1000}$, $P(X = 0) = \frac{972}{1000}$.
          \item Expected value:
            \[
              E[X] = 200 \cdot \frac{1}{1000} + 20 \cdot \frac{27}{1000} + 0 \cdot \frac{972}{1000} = 0.56.
            \]
        \end{itemize}
    \end{itemize}

  \item \textbf{Properties of Expected Value:}
    \begin{itemize}
      \item Units of $E[X]$ match the units of $X$.
      \item $E[X]$ may not be a value in the range of $X$.
    \end{itemize}

  \item \textbf{Advanced Examples:}
    \begin{itemize}
      \item \textbf{Geometric Distribution:}
        \begin{itemize}
          \item PMF: $P(X = k) = (1 - p)^{k-1}p$ for $k = 1, 2, \dots$.
          \item Expected value:
            \[
              E[X] = \frac{1}{p}.
            \]
        \end{itemize}

      \item \textbf{Poisson Distribution:}
        \begin{itemize}
          \item PMF: $P(X = k) = \frac{\lambda^k e^{-\lambda}}{k!}$ for $k = 0, 1, \dots$.
          \item Expected value:
            \[
              E[X] = \lambda.
            \]
        \end{itemize}

      \item \textbf{Binomial Distribution:}
        \begin{itemize}
          \item PMF: $P(X = k) = \binom{n}{k}p^k(1 - p)^{n-k}$ for $k = 0, 1, \dots, n$.
          \item Expected value:
            \[
              E[X] = np.
            \]
        \end{itemize}
    \end{itemize}

  \item \textbf{Key Techniques:}
    \begin{itemize}
      \item Use summation formulas for uniform and geometric progressions to simplify calculations.
      \item Understand difference equations for summation proofs.
    \end{itemize}
\end{itemize}

\section*{Simplified Explanation}

\textbf{Expected Value:}
The "average" value of a random variable, found by summing each possible value weighted by its probability.

\textbf{Examples:}
- Bernoulli($p$): $E[X] = p$.
- Uniform(1 to 6): $E[X] = 3.5$.
- Geometric($p$): $E[X] = \frac{1}{p}$.

\textbf{Why It Matters:}
Expected value helps understand long-term outcomes and guides practical decisions (e.g., betting strategies).

\section*{Conclusion}

In this lecture, we:
\begin{itemize}
  \item Defined expected value and its practical significance.
  \item Demonstrated calculations for common distributions.
  \item Highlighted techniques for simplifying summations.
\end{itemize}

Expected value bridges probability and real-world applications, providing insights into the central tendency of random variables.

\end{document}
